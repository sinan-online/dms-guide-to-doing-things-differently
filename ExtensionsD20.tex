\documentclass[twocolumn]{dndbook}

% Compile using UTF-8
\usepackage[utf8]{inputenc}

\usepackage{xkcdcolors}

\colorlet{dirtyorange}{xkcdDirtyOrange}

\usepackage{mdframed}

\usepackage{hyperref}

\hypersetup{
	colorlinks=true, % Whether to color links (a thin box is output around links if this is false)
	%hidelinks, % Hide the default boxes around links
	urlcolor=dirtyorange, % Color for \url and \href links
	linkcolor=black, % Color for \ref/\nameref links
	citecolor=dirtyorange, % Color for reference citations like \cite{}
	hyperindex=true, % Adds links from the page numbers in the index to the relevant page
	linktoc=all, % Link from section names and page numbers in the table of contents
}

\newenvironment{emphasisParagraph}{
	\begin{quote}
	\begin{mdframed}[
		topline=false,
		bottomline=false,
		rightline=false,
		skipabove=\topsep,
		skipbelow=\topsep,
		linecolor=xkcdBottleGreen,
		linewidth=3pt,
	]
	\em
}{
	\end{mdframed}
	\end{quote}
}

\usepackage{titling}
\newcommand{\subtitle}[1]{%
  \posttitle{%
    \par\end{center}
    \begin{center}\large#1\end{center}
    \vskip0.5em}%
}

\usepackage{biblatex}
\addbibresource{ExtensionsD20.bib}



\begin{document}

\title{The DM Guide to Doing Things Differently}
\subtitle{(or The Book of Diverse Challenges)}

\maketitle

% The DM's Guide to Doing Something Different
% The DM's Guuide to Designing Non-Violent Challenges
% The DM's Guuide to Designing Diverse Challenges

\chapter{Introduction}

This book offers mechanisms for a diverse set of challenges.
I wrote this book, because I did not want my games to be stuck in a repetitive pattern.
The traditional game follows the following pattern: three easy combat encouters, party is statistically guaranteed to survive unless they active try to get themselves killed.
Two traps. Dungeon crawl. Boss battle, a few of the player characters may die.
Characters met in a tavern, don't ask why they were in the tavern or why they decided to walk out together.
Players and the ``bad guys'' are both trying to kill each other, but we do not know why exactly.\par

Even the pulpest pulp action movie has mor challenges.
Action movies will have conflicts with nature,

\begin{emphasisParagraph}
	Fantasy literature has moved beyond the trope of a band of people going and killing a boss.
	Role playing game mechanics have not, and it's getting boring.
	How to make it more fun?
\end{emphasisParagraph}


Fantasy literature itself has long moved beyond this formula.
We do not watch Game of Thrones or Witcher just for the boss battles --- combat encounters are only a small portion of the entire set of challenges.
Up until the boss battle, we watch the characters deal with social drama, political intrigue, conflicts with nature and internal conflict.
Seeing characters in these diverse set of encounters makes us relate to them, and care about them.
I wanted to capture all of this in my games, so I started looking for mechanics for a diverse set of challenges.\par

The traditional challenges in fantasy games, are combat, typically to death, and traps and puzzles.
Typically, games seem to be stuck in a false dichotomy of roleplay vs combat.
However, fantasy captures readers when they can relate to the characters in some way.
To that end, this book strives to provide ways to create encounters to make our characters relatable.\par

\begin{emphasisParagraph}
This book adds mechanics for a diverse set of non-combat challenges and an option to bulid diverse \& relatable characters.
\end{emphasisParagraph}

Most non-combat challenges in roleplaying games are resolved through either roleplay, or one roll of the die.
Roleplay is fun. However, I find that players have more fun if there is a chance that they will fail.
This means that we involve the dice to simulate stakes.
Therefore, the guiding principle is that if there is going to be roleplay, it should be accompanied by roll-play.\par

\begin{emphasisParagraph}
	Guiding principle 1: if there is going to be a challenge resolved through roleplay, combine this with roll-play mechanics.
	The probability of failing makes it more fun.
\end{emphasisParagraph}

At the beginning of each chapter, you can find why you might want to use the mechanics in the chapter.
If those are not of interest to you, feel free to pass them by.
At the end of each chapter, are playtest notes.
This keeps me honest --- do the rules and mechanics work as intended?
They may also be useful for inspiration.\par

\begin{emphasisParagraph}
	Guiding principle 2: Give players multiple paths for resolution.
\end{emphasisParagraph}

One difficult that I personally had with my first D\&D campaign was how NPCs were able to arbitrarily ``do stuff''.
Necromancers will create armies of undead, but my necromancer PC cannot.
There may be flying cities in the setting, but the best I can do with a spellcaster is the \emph{fly} spell.



\begin{emphasisParagraph}
	Guiding principle 3: If an NPC can accomplish \emph{X}, there should be a way for PCs to accomplish \emph{X} as well.
\end{emphasisParagraph}



Feel free to use each chapter separately, or altogether. Enjoy.




\chapter{d20 Real Combat}
\label{chap:d20_real_combat}

\begin{emphasisParagraph}
	\paragraph*{Consider using this mechanic if} any of the following is true:
	\begin{enumerate}
		\item you want to run a survival themed session where party members may die
		\item you want to add realism to combat to get your players to strategize and plan ahead of combat
		\item you want to dissuade your players from jumping to combat-based solutions
	\end{enumerate}
\end{emphasisParagraph}

% TODO: First edit

\subsection{Hit Points Break Immersion}

Your player overcomes multiple challenges through creative solutions and makes the rolls.
They are in the bedroom of the local tyrant. They want to slit their throat and kill them.
They can make an roll with advantage, pass all checks, but they are still unlikely to dispatch the enemy with one blow.
It is almost as if all of that protection was for nothing - these ``boss'' NPCs are almost impervious to anything.
However, if you repeatedly hit them with the pointy end, eventually they will succomb and fall.\par

My point is that this breaks anyone's immersion.
Computer games allow for stealthy play --- we should be able to do so in a game.
However, the hit point mechanic is one of the oldest and most established rules.
There are multiple reasons why they stood the test of time --- almost anything else is too difficult to play.
Instead, we will make a compromise.
We will use the hit point mechanic, but make a few changes so that violence is realistic.

\subsection{d20 Real Combat Rules}

Make three changes to the usual hit point mechanic:
\begin{itemize}
    \item No level bonuses to hit points.
    \item Hit points are regained fully at the end of each round.
    \item Potion-drinking is a bonus action
\end{itemize}

\paragraph*{No level bonus to hit points}
The first amendment means that humans, humanoids and all players medium sized races have only one hit die worth of points, based on their class and constitution.
Larger and magical creates still have higher hit points, and will be extremely difficult to kill.

\paragraph*{All hit points are regained fully at end of each round}
The second rule means that unless an attack or multiple attacks reduce a character to 0 hit points, they do not die.
This makes it easy to kill player characters, but really difficult to kill large or gargantuan monsters who have lots of hit points thanks to their size.\par

The purpose of these changes is to to incentivize players to avoid combat if possible, and challenge them to find other solutions to problems.
If the combat does take place, we incentive them to think it through and strategize carefully.
Once it does take place, it will feel epic, because each round is going to be high stakes.
Alternatively, it will feel satisfactory to players, because it will be an very well-planned ambush where the enemy does not even get a chance to fight back.
If the plan fails, or if the players themselves are ambushed, they will be compelled to use potions, powers and inspiration points to save themselves.
This means that a high-level fighter can easily die from a goblin ambush, and a dragon is so difficult to kill that requires either a group of characters that can deal very large amounts of damage, or an archer batallion.\par

If an attack hits, but does not deal enough damage combined with other attacks in the same round to bring the HP to 0, this means that the target survives.
You may narrate this as dodging, or a hit being deflected by armor, or a scratch or a bruise.
Alternatively, if the characters drink a potion, it may mean that they were wounded, but the wounds healed magically.
The final interpretation is up to the GM.
When the players drop down to 0 HP, this means that they are mortally wounded, and the death saves start as usual.

\paragraph*{Potion drinking is a bonus action} To give players more options, we make potion-drinking into a bonus action.
Unless the potions are consumed within the round, they lose their usefulness under these changes.
If the characters have their potions on a belt or in an accessible location, they can drink the potion as a bonus action.
However, taking a potion out of a backpack is a full action.\par

\subsection{Examples}

\paragraph*{Example: Ambushing spellcasters} This mechanic makes battles with mages deadly.
If a party attacks a spellcaster with no plan in mind, the first PC that the mage casts \emph{Magic Missile} on is almost certain to lose consciousness.
This adds realism and high stakes, it is similar to a party attacking a human with a modern automatic handgun.
It is also in line with most of fantasy literature as sellcasters become very powerful adversaries.
What can a party do? They may try to obtain protection, and then trick the spellcaster into attacking the party member with the protection, while the rest of the party tries to ambush them within one round.
Alternatively, healers may simply hold their action to heal the hurt party members.
They can distribute potions to all party members, have them ready and take them as bonus actions.
Finally, they can decide to sneak up to the spellcaster to prevent them having a chance.
One way or the other, the spellcaster needs to be dealt with in one or two rounds, otherwise the party risks losing a character.
This high-stakes and fast combat is more reminiscent of real-life combat, and is intended to increase both the climactic tension, and increase the fun through planning for the climactic battle.\par



\subsection{Playtest Notes}

I have not playtested this part at all! If you use them, please contact me to add, I would appreciate that a lot!


% TODO: Add an example where a troll is ambushed and killed with a combinaion of burning hands, a fighter hitting with a sword, and a backstab.
% TODO: Add an example of a combat where then same troll gains the initiative and attacks.


\subsection{Future Work}
I am planning to add mechanics for being bruised or wounded.


% TODO: Add a condition called "Bruised" where you get 1d4 to intimation, substract 1d4 from all other Charisma checks.
% TODO: Add a condition called "Wounded" where you substract 1d4 from all "physical" checks.



\chapter{Specializations: d20 + d4}
\label{chap:d20_plus_d4}

\begin{emphasisParagraph}
	\paragraph*{Consider using this mechanic if} you want to
	\begin{enumerate}
    \item reward your players for writing detailed backstories
    \item add some character quirks without breaking the balance of the game
    \item allow creation of classes characters (diplomats, sages, merchants, or even commoners) while keeping them relevant and sufficiently powerful.
    \end{enumerate}
\end{emphasisParagraph}

\paragraph*{TL;DR} Add a \emph{specialization die}, like the Bardic inspiration or \emph{Bless} to the rolls.


SRD 5e is great in its simplicity, but makes it difficult to customize characters with specializations.
Consider checks to find out if a character knows about the climate of a country that they did not visit,
or international politics, or the main trade routes in a region.
The DM will likely default to Intelligence (History) checks, even though these are not related to history.
In contrast, previous editions of SRD would have had specific knowledge checks, but then these were rarely useful,
so it was not rewarding to invest in these at character creation.\par

To get the best of both worlds --- in depth character creation and simplicity --- we introduce ``specializations''.
If, a character spends more than 8 hours a day for at least 5 years in a profession, on a hobby or an area of study,
they get the ``specialization''.
This gives them a $1d4$ die to proficiency checks, stacking with \emph{Bless} and Bardic inspiration.\par

\paragraph*{Preventing Critical Failures}
Unlike \emph{Bless} and Bardic inspiration, this die can prevent a critical failure:
So long as the specialization die is not $1$, the DM can present the situation as a ``close save''.
The roll can still be a failure, though, if the total does not pass the DC.\par


\paragraph*{Coherent Backstory} Players who want to have these specializations,
need to account for all of the years in their backstory.
For example, if a character spent at least five years working as a tailor prior to adventuring,
they can get the ``tailoring'' expertise.
They need to spend at least 8 hours a day to be a specialist in these area, meaning that they have little time for anything else.
There is no way that they apprenticed with a carpenter or a blacksmith since they already
had a day job, so they cannot get these specializations from the same years.
This also means that they are not a noble, and they are a character who had to earn money to make a living\par

To make things easier, we can add specializations to a character \emph{as they become relevant}.
Unlike the usual proficiencies, specializations will come up less frequently, and can be developed
retrospectively.\par

\paragraph*{Flexibility, balance and fun}
Suppose that the previous ``tailor'' character's player also wants them to have a specialization in ``Animal Handling --- Horse''.
The player might claim that the character spend time riding and taking care of their horse.
This is probably a difficult thing to achieve: if they were a tailor, they were probably in the city, how did they find a place to keep the horse?
How did they pay for a horse --- owning and maintaining a horse is a luxury for a city-dweller.
How did they even find the time and energy?
One way that the DM may go is to rule that this is inconsistent with the backstory and refuse.
However, another way to deal with this is to direct these questions to the player, and watch a more colorful character grow.
Maybe, they rode their horse beginning all the way at age 15, for over ten years, at least a few hours every night.
(So that the total hours match.)
Maybe they had to move to the city and leave their beloved horse behind\ldots
It can come together to make a coherent whole.
The whole purpose of ``specialization'' is to make character building fun and go beyond the usual fantasy roles.\par


\paragraph*{Long-lived races} This rule intentionally gives long-lived races an advantage.
This is in parallel with fantasy literature and common sense: a hundred year old elf is
bound to have some advantages over their 20-year-old human companions, although perhaps not in adventuring.\par

\begin{DndTable}[header=Specializations]{X}
	\\	Carpentry
	\\	Massage
	\\	Roof Tiling
	\\	Shoemaking
	\\	Sailing
	\\	Baking
	\\	Boat Building
	\\ 	Brewing
	\\	Winemaking
	\\	Firefighting Techniques
	\\	Stonemasonry
	\\	Swimming
	\\	Animal Handling --- Specific Animal Species
	\\	Singing
	\\	Comedy / Joking
	\\	Musical Instrument --- Specific Instrument
	\\	Embroidery
	\\	Knitting
	\\	Painting
	\\	Drawing
	\\	Cooking
	\\	Algebra
	\\	Anatomy --- Humanoids
	\\	Anatomy --- Aberrations, Large
	\\	Anatomy --- Aberrations, Small
	\\	Anatomy --- Aberrations, Medium
	\\	Anatomy --- Aberrations, Huge
	\\	Anatomy --- Beasts, Large
	\\	Anatomy --- Beasts, Small
	\\	Anatomy --- Beasts, Medium
	\\	Anatomy --- Dragons
	\\	Anatomy --- Fey, Tiny
	\\	Anatomy --- Fey, Small
	\\	Anatomy --- Fey, Medium
	\\	Anatomy --- Fey, Large
	\\	Anatomy --- Fiends, Large
	\\	Anatomy --- Yugoloth
	\\	Anatomy --- Monstrosities, Large
	\\	Anatomy --- Monstrosities, Huge
	\\	Anatomy --- Monstrosities, Medium
	\\	Anatomy --- Giants
	\\	Astronomy --- Earth
	\\	Astronomy --- Faerun
	\\	Astronomy --- Eberron
	\\	Astronomy --- Oerth
	\\	Bookkeeping / Accounting
	\\	Botany / Herbes
	\\	Alchemy / Chemistry
	\\	Philosophy
	% \item Politics
	\\	Theology --- Specific religion
\end{DndTable}



\paragraph{*Adding new specializations} Here is the rule of thumb:
Specializations have to be one of the following:
\begin{enumerate}
    \item something learned through an apprenticeship or receive on-the-job-training
    \item something learned through a 100-level university course, or an equivalent basic course in the campaign setting
    \item something learned or developed through night classes
\end{enumerate}
``Specialization'' has to be more specific than the usual proficiencies in the fifth edition,
that is the point. However, if you go more specific than a 100-level university course, then
it becomes too specific to be relevant for most game sessions. This rule of thumb
ensures that you stay in the

\paragraph*{Increasing the specialization die}
Specializations get better only through spending time on them.
Use the following table to increase the die.
\begin{DndTable}[header=Time Spent \& Die]{XX}
Time Spent	&	Specialization Die \\
5 years		&	1d4 \\
10 years	&	1d6 \\
50 years	&	1d8 \\
100 years	&	1d10 \\
500 years	&	1d12 \\
\end{DndTable}

This reflects the diminishing returns to specializing in one area.
At the same time, some non-human races or ancestries can be used to create in depth characters.
This can also be used to make some weak NPCs really powerful in one specific area,
for example, you could have an elven botanist who spent an entire lifetime with plants, and is excellent at brewing potions.\par

% TODO: Give examples with extra rolls.
% TODO: Gives the real-life examples of librarian, merchant, and fashion designer / socialite.


\section{Optional Rule: Local Specialization}

% TODO: Finish the ``local specialization'' rules.

\section{Optional Rule: Age \& Wisdom}

\section{Optional Rule: Forbidden Knowledge \& Power}

% TODO: The essence of the ``forbidden bonus'' is this: Give them a bonus die, but tell them that a $1$ from the bonus die is also a Nat 1. For dice The idea is this: they freely get this extra boost, and it is easy to obtain, but something nefarious is likelier to happen.
Almost all cultures have an age level where an individual is expected to be ``wise''.
Consider giving an advantage to all Wisdom rolls over a certain age, both to players and to NPCs.
In my games, I consider this age to be 60.
This will give elven and even dwarven characters a serious advantage (pun intentional).
However, it is a more realistic depiction of these characters, and is a way to recapture the position of elves and dwarves in relation to humans.

% TODO: Review the age & wisdom wording.

\chapter{Dramatic Mechanics}

\begin{emphasisParagraph}
	\paragraph*{Consider using these mechanics if} you want to give challenges other than combat to your players.
\end{emphasisParagraph}

There are a host of dramatic challenges in literature and movies.
However, the two climatic challenges in role playing games are puzzles and combat challenges, typically fought to death.
Whenever there is another challenge, this is resolved through roleplay, and typically, one proficiency roll.
This chapter aims to give combat-like mechanics for other types of challenges.\par

Most of the challenges come in two stages: preparation and the actual challenge.
This is intended to increase tension leading up to a climactic resolution.\par

\section{Exams}

It is uncommon to have exams in fantasy settings, but being student, or needing to pass exams is a common experience that almost all players will relate to.\par

There are multiple reasons to have an exam as part of an adventure. Here are some:
\begin{enumerate}
    \item exams may be a good alternative to the ``meet in a tavern trope''
    \item an exam may be required to be members of an organization
    \item adventurers may be required to pass an exam to gain access to institutions of learning
\end{enumerate}


\paragraph*{Preparation}

Here is the basic rule for studying for an exam:
Step 1: Roll Insight (Wisdom) with a DC of 11.
Step 2: If you pass, you get a ``preparation bonus'' of +1. However, it does not end there. You can now roll against a difficulty of DC 12. If you pass this one, your ``preparation bonus'' increases to +2.
Step 3: Repeat the roll with incrementing DC until you fail. For example, if you failed at DC 15, your ``preparation bonus'' will be +4.

In case of a critical failure, decrement the bonus by 1. In case of critical successful, get advantage on the next roll.
Each roll takes 1 hour of time in the game, and exhaustion rules may apply. You can prepare for an exam only during the week leading up to the exam.\par

If the player did not go to any one of the classes, all rolls are at a disadvantage. If the player skipped some classes, start at the 4th roll and roll disadvantage afterwards.\par

Preparing cheat sheets and cheating apparatus is another alternative.
Ask the player what exactly they are doing.
Depending on the answer, get them to roll either a proficiency check with a Forgery Kit, Calligraphy Kit, or a perhaps Cartographer's Kit (Intelligence).\par

Alternatively, Artificer characters may prepare infusions, or spellcasters may prepare spells, or all classes may attempt to hide an item with an Intelligence check.
A disguised \emph{Helm of Telepathy} or a prepared \emph{Detect Thoughts} can be very useful.\par

% TODO: Forgery Kit is from Xanathar's Guide to Everything. Helm of Telepath is DM's guide.

Things can happen during preparation. If desired, here are some ideas and a table.
\begin{DndTable}[header=Random Event While Preparing for an Exam]{rX}
	Roll	&	Event \\
	1		&	A friend invites for a night out \\
	2		&	A rival comes over and brags about what they did for study. \\
	3		&	Exam questions are rumoured to be available. Pay to get the questions, or continue studying? If real, going over the questions will give an adventage on the day of the exam. \\
	4		&	Player gets a moment of epiphany. Advantage on the next study roll.
\end{DndTable}


\paragraph*{Day of the Exam}

An exam will have three or four questions or sections. Roll Investigation (Intelligence) against DC 15 for each question, and add the ``preparation bonus'' from earlier.\par

If the characters are trying to smuggle in cheating equipment, such as magic items or cheat sheets, succesfully hiding them would require a Deception (Charisma) or a Sleight of Hand (Dexterity) check.\par
Use the following as a guide to determine the DC:\par
\begin{DndTable}[header=Smuggling Cheat Sheets into the Exam]{Xl}
	Exam	&	DC \\
	No search, just a midterm with little oversight	&	5 \\
	Midterm with some tired assistants	&	10 \\
	Finals with multiple vigilant assistants	&	15 \\
	High-stakes exam & 20 \\
\end{DndTable}

If desired, you can roll a ``luck'' $d20$ and determine the result as the DC. Perhaps the proctors decided to be extra vigilant on that day\ldots\par

Cheating will typically require a Sleight of Hand (Dexterity) check, typically against the passive perception of the nearest proctor. (Use the DC from table above as the passive perception, or roll a $d20$ to determine. If they are unlucky, the proctor will be particularly vigilant.)I
If copying from another person, roll Investigation (Intelligence) against DC 15 to see \emph{if they got it right}.\par

% TODO: Link to Eberron Papers Please

\paragraph*{Aftermath}

Consider giving a bonus if the players pass a high-stakes exam without cheating --- see \nameref{chap:d20_plus_d4}.

\subsection{Playtest Notes}

I used this as a session 0 tactic. The players met each other. I gave them an exam subject, and asked them why they were taking the exam. This cued them to create characters with more detailed backgrounds.
I also clearly gave them the option to study or to cheat, which made it more fun by giving them a choice. The study mechanic was a bit too complex and got in the way of enjoyment while explaining.\par


\section{Bridging the Language Gap}

\begin{emphasisParagraph}
	In real life, there is no ``common'' language.
\end{emphasisParagraph}


``Common'' is a good simplifaction, but removes a lot of the chaillenges that can be used in gameplay.
Misunderstandings across languages is a common thematic element and cause for dramatic tension.
The table below provides a mechanic to play out language comprehension.\footnote{The rules here are inspired by the wonderful work \href{https://www.dmsguild.com/product/262786/Languages-of-Eberron}{Languages of Eberron}.
It is pay-what-you-want, please consider using it if you are into linguistics and want to use that in a campaign in Eberron.}\par

\subsection{Language Proficiency}
All languages come with a proficiency level: ``Basic'', ``Intermediate'', ``Advanced'' and ``Native''.

At character creation, choose one or two ``native'' languages in a way that makes sense with the charcater backstory.
These are languages that you either spoke at home, or spoke during your teenage years with your peers.\par

Based on the players handbook, number of languages that a character can speak is a trait based on race.
All characters begin with two languages, except for high elves and half elves, who speak three.
To keep it consistent, assume that a character has 7 points to allocate, or 10 if they are half-elf or high elf.
Allocate the points based on the following table.
\begin{DndTable}[header=Language Proficiency Levels]{Xrr}
	Proficiency Level	&	Point Cost & DC to Understand and Be Understood in ``familiar'' language\\
	Basic	&	1 & 15\\
	Intermediate	&	2 & 10\\
	Advanced	&	3 & 5\\
	Native & 4 & 0\\
\end{DndTable}

During the game, understanding a ``familiar'' language is a DC 20 Intelligence check.
Each level in the language proficiency drops the DC by 5.
So if you are an ``advanced'' speaker of Elvish, but it is not your native language,
you can roll againt DC 5 to understand what people are saying.
It is only when you reach ``native'' proficiency that the DC drops to 0.\par

Depending on the setting, ``familiar'' may mean either ``same language family'' (like Indo-European languages in our world),
or just the regular languages: elvish, dwarven, halfling and gnome. Languages that are not ``familiar'',
either because they are unfamiliar to the characters, such as Dranonic, or languages that are from other language families add +5 to the DC to understand,
and another +5 if they are using another script, to a grand total of DC 30.
This reflects that it is almost impossible to understand language with a different writing system or speech in a language that shares almost no vocabulary.\par

If using this rule, \emph{Comprehend Languages} adds +20 to the roll to understand.\par

\subsubsection{Optional Rule --- Reading \& Illiterate Characters}
The rules for understanding and communicating in a language are same for written and spoken.
This is to keep it from becoming too complex, but if you like, you can make the roll at an advantage if trying to understand written language, or communicate in written form.
This advantage reflects the ease that comes from being able to see the entire passage at once.
However, if the written language is in another script or form of writing, make the roll at a disadvantage.

\subsubsection{Optional Rule --- Pidgin Languages}
Pidgin language is a grammatically simplified mode of communication between two languages.
If using a pidgin languge in the campaign, add +5 bonus if the character has proficiency at least one of the languages.



\subsection{Playtest Notes}

I playtested this in a campaign that involved goblin communication to two encounters. My players loved it.\par

If there is no combat situation going on, it is best to make one roll at the beginning of a dialogue.
Ask for a second roll once something important comes up, such as the a critical piece of information.
Otherwise, the rolling gets tedious very quickly.

\subsection{Future Work}





\section{Job Interview}
\label{sec:job_interview}

Player can be taking the interview, or interviewing an NPC --- this mechanic allows for both.\par

The job will require at least one skill --- for the sake of example, let's say that it is proficiency with Artisan's Tools.
The interviewer determines a DC. If the interviewer is an NPC, this can be played over in three rounds where the DC keeps increasing.
The character being interviewed can choose to roll Deception (Charisma) or the skill that is required for the job.
If they choose to roll the required skill, they need to beat the DC. However, the interviewer also needs to roll Insight (Wisdom) against the same DC.
If they both succeed, then the interviewer is convinced that the character is skilled.
If the interviewer rolls lower, they are not sure.
If interviewee rolls Deception (Charisma), the interviewer will roll Insight (Wisdom) in contest.
If the interviewer rolls higher, they find out that the character is deceiving them.
If they intrviewer rolls lower than the DC they set, they are not sure about the skill level.\par

This mechanic ensures a few things: First, the interviewee has a choice, which makes it more fun for players to play.
Second, if the player is the interviewer (for instance, if they are hiring a henchman or a staff for their business)
the player will always be required to roll Insight (Wisdom), which will keep them uncertain about whether they were deceived or not.\par

\subsection{Variant: Deception}

The same kind of mechanic can be used whenever a player wants to understand the intentions of an NPC.
Say that the DC is 12. If the player rolls less than 12, they do not know the intentions of the NPC.
If they roll higher than DC 12, they think that they know the intentions of the NPC, \emph{unless the NPC rolled a higher number on Deception (Charisma).}
This will keep players second guessing whether they understood the intentions of an NPC correctly even when they win.\par

\subsection{Playtest Notes}

I have not playtested this mechanic yet, and I would love to hear feedback.

\subsection{Future Work}

I am not planning to update this mechanic at the moment.


\section{Running a Small Business}

\begin{emphasisParagraph}
	A shop that the player values is a great way to use hooks into a large number of adventures.
	You know how it is the merchants that are often quest givers?
	It will hit closer to home to your players if they are the shop owner.
	\paragraph*{Consider using this mechanic if you}
	\begin{itemize}
    \item think that the rule from the DM's guide are not fun enough,
    \item want to create bonds with hired NPCs and the players, to use later as hooks,
    \item want to create other hooks for adventures.
    \end{itemize}
\end{emphasisParagraph}


Running a business is a ``downtime activity'' in the DM's guide.
The rules involve set costs and a $d100$ table.
The probability of turning is a profit is not impacted by the player's decisions or the character's abilities.
In real life, a small business is usually complex.
Revenue is not guaranteed, even though costs are fairly standard.
Some people manage to make it work, and other do not.
This is a lost opportunity in gamifying, so here are some new rules.\par

At the end of each month:
\begin{enumerate}
	\item Deduct the costs
	\item Roll a skill check to see if the shop generates Revenue
	\item If successful, roll a ``revenue die'' to determine how much.
\end{enumerate}

\subsection{Determine the Costs}
A shop incurs costs monthly.
According to the DM's Guide, the standard cost is 2gp per day, and half of that goes to a skilled employee.
Use the following table to offer alternatives to the player.\par

\begin{DndTable}[header=Shop Costs]{Xll}
Cost (per day)	& Revenue Cap (per day) \\
2 gp	&	4 gp \\
3 gp	&	8 gp \\
4 gp	&	12 gp \\
5 gp	&	16 gp \\
6 gp	&	20 gp \\
7 gp	&	24 gp \\
\end{DndTable}

In this table, the first option is the standard given in the DM's Guide.
You can interpret the next two options as better locations, better foot traffic, or a richer neighborhood.
This increases the total sellable market, allowing for higher revenue.
Note that the costs are incurred monthly.
For example, in the standard option, player pays $2 x 30 = 60$ at the end of each month.\footnote{Adjust to the setting as needed for consistency, for example, Eberron has 28 days in month.}\par

The salary of the skilled employee is included in these costs, and is 1 gp, as per DM's Guide. In other words, the cost increase comes from the increase in rent.\par

\subsection{Determine the Revenue}
If a character decides to spend the month at a shop, roll either one of the three checks:

\begin{DndTable}[header=Skill Check to Generate Income]{lX}
Check	&	Conditions \\
Tool proficiency (Dexterity)	&	... if the shop is an artisan's or craftperson's shop and they are doing honest work. \\
Persuasion (Charisma)			&	... if they are doing trade, like a merchant or a general goods store. \\
Deception (Charisma)			&	... if they are running a scam or outright trying to deceive customers.	\\
\end{DndTable}

This replaces the $d100$ table in the DM's Guide, under ``Running a Business''.
If the character does not spend time at the shop or business, the DM rolls as the hired help.
Either create the employee as an NPC or simply use +5 as the bonus.
So long as the skilled employee is working, the player can roll at an advantage.\par

If the check passes, roll the ``revenue''. This is a lot like the ``damage roll''.
The monthly revenue is going to be the outcome of the die, but will be capped by the ``Revenue Cap'' in the table above.\par

% TODO: Explain nat 20 and nat 1.

\begin{DndTable}[header=DC \& Revenue Die]{Xrrr}
DC	&	Revenue Die \\
5	&	$1d4$\\
10	&	$1d6$\\
15	&	$1d8$\\
20	&	$1d10$\\
25	&	$1d12$\\
\end{DndTable}

For each month, roll the proficiency, against the DC chosen by the player.
If passed, roll the ``revenue'' die, cap by the location, and multiply by the number of days in the month.\par

\subsection{Events \& Hooks}

A shop that the player values is a great way to use hooks into a large number of adventures.
It will be way more impactful for the player if it is their own income that got distrupted by bandits,
or if an employee NPC that they have had for a while gets kidnapped, or betrays them.
To find a new employee, they need to interview. See the \nameref{sec:job_interview} section;
this also provides a mechanic for the player being deceived about the skill or th intention of the NPC.\par


Use the following either as a random table whenever you want to introduce a new encounter, or as inspiration to connect to adventures.
If the NPC working for the shop is gone, the cost per day drops by 1 gp, but the shop does not bring in revenue unless the player steps in or finds a replacement.
If the shipments do not come in, the shop cannot generate any revenue.\par

\begin{DndTable}[header=Events]{rX}
	Roll	&	Event or Hook \\
	1	&	The NPC working for the shop disappears. (They are kidnapped.)\\
	2	&	The NPC working for the shop rolls their relevant skill check against DC 20. If they win, they find another job and leave immediately.\\
	3	&	The NPC working for the shop leaves abruptly. They have a note saying that they are going to a relative's funeral, and will be back next month. Roll $1d4$, if the result is 1, they do not come back, otherwise, they come back the next month.\\
	4	&	The NPC working for the shop rolls Deception (Charisma) against the PP of the player. If they succeed, tell the player that the revenue is 1 less than the ``revenue die''. They will try again next month until they fail. If they fail, the shop owner catches them. \\
	5	&	Increased competition --- either a new store or an existing store engaging in a price war. The DC is increased by 5, and will remain so unless competition is removed.\\
	6	&	If there is increased competition, it ends. Otherwise, nothing happens.\\
	\end{DndTable}

\subsection{Optional Rule: Detailed Costs, Leases \& Investment}

% TODO

\paragraph*{Potential hooks \& extra challenges}
% The shop is attacked and merchandize stolen.
% Their shipments stop.
% The player may try to become a monopoly. (All checks pass automatically without needing to roll.)
% Opportunities to make extra money by selling unique items / special orders.


\subsection{Playtest Notes}

\section{Networking}

% TODO: Roll Cha to find out if people remember you. DC based on time passed.

\section{Battle Preparation}

In almost all of the D\&D sessions that I played in various settings, the party went in blind to boss battles.
This is in contrast with a lot of what I see in fantasy literature.
Witchers seem to have a lot of theoretical and practical knowledge about beasts.
In the Witcher games, you have to pick the right potions, and get the right ``signs'' to use based on the monster.
In multiple computer games based on various editions of the D\&D rules, you would need potions or spells based on specific vulnerabilities.
In almost all of the computer games with boss battles, you have to save and load multiple times before coming up with a strategy.
What I am trying to get at is: preparing for the battle can be part the fun.\par

\begin{emphasisParagraph}
	\paragraph*{Preparing for the battle can be part of the fun.}
\end{emphasisParagraph}

\paragraph*{Cue the players}
As GM, consider using the following sentence: ``you will all die if you go into the battle without preparation''.
If possible, have an NPC make this very clear, or if you can, kill an NPC that they know to be strong and powerful.
Honestly prepare a boss that is twice as powerful as usual, or use the mechanic from \nameref{chap:d20_real_combat}.\par

\paragraph*{Doing Research}

For each ``piece of information'' about a creature, players roll Investigation (Intelligence) DC 20 + Modifiers.

% For stats, roll the stat of the monster and add it to the DC. That's the modifier.
% To understand the save against an ability, roll the save. That's the modifier. If they pass, tell them tha
% To understand the

% Libraries give bonus. Larger libraries give better bonus. Internet gives a lot of bonus, $d10$. An arcane library gives $d12$. Access to forbidden knowledge and generative AI gives $d20$, but either one of the dice count as Nat 1.

\subsection{Examples}

\subsection{Playtest Notes}

% TODO: Bonuses - learning real name of a demon
% TODO: Bonuses - learning the backstory/identity of a ghost, getting a relevant item, giving a proper burial.


\section{Financial Management \& Investment}
% Start with the Dutch tulip market craze
% TODO: Medieval era trading investments, like Janissaries or Arabian Nights.
% TODO: "safe" vs "risky" investments
% TODO: Modern era market investments.


\section{Social Endeavors}
\subsection{Making An Impression}
\subsection{Public Speeches}
\subsection{Dating}

\chapter{External Conflict}
\section{Conflict Against Other Characters}

\subsection{Chases}
% TODO: Just put in the regular D&D mechanic
% TODO: Explain aerial combat specific issues
% TODO: Give the example of the aerial combat from my game.
\subsection{Racing}
\subsection{Sports \& Competitions}

% TODO: Otso Borno from this link, it is like Sumo: https://www.reddit.com/r/dndnext/comments/4jr6ly/comment/d39842b/
% However, I want to turn this into a coordination game: you choose STR or DEX. If you choose DEX when opponent chooses STR, you roll at an advantage. But if you both choose DEX or STR, a penalty happens, or something like that. You should also have an optional "rolling for insight" rule.
% Optional rule: Make it a drinking game.
% Optional rule: need three consecutive strikes...

% TODO: Billiards https://www.dndbeyond.com/forums/d-d-beyond-general/story-lore/22670-in-game-minigame-ideas

\subsection{Gambling \& Tavern Games}

There are many homebrew resources for tavern games.
The games here are based on the excleent Reddit post by u/eryan64, ``\href{https://www.reddit.com/r/DnDBehindTheScreen/comments/fn6tng/a_collection_of_tavern_games/}{A Collection of Tavern Games}''.
I am striving to create realism, and give the player a choice.

\subsubsection{Blackjack}
% https://www.reddit.com/r/DnDBehindTheScreen/comments/fn6tng/a_collection_of_tavern_games/

\paragraph{Cheating}
\subsubsection{Slots}

\paragraph{Slots: Realistic Odds and Payoffs}

\paragraph{Slots: Realistic Odds and Payoffs}

\subsubsection{Gathering Information while Playing}

\subsubsection{Dice Poker}

% TODO: Dice poker from The Witcher https://witcher.fandom.com/wiki/The_Witcher_dice_poker

\subsubsection{Drinking Games}

\subsection{Dramatic Standoffs \& De-escalation}

% TODO: Rewrite in a simplified manner.

Encounters do not always need to end in violence.
In the Western genre, there is oftentimes a ``standoff'' or ``showdown'' right before a gun battle.
This is a chance for one of the parties to back off, intimidated by their odds.
You could resolve such a standoff as a simple intimidation roll, or you could turn it into a set of decisions.\par

Just like in the Western movie trope, the characters can choose to brandish or ready their weapons.
If they brtandish their weapons, this gives them an advantage in Intimidation (Cha) rolls.
I suggest 20 + CR of the highest ranking enemy + number of enemies as the DC or 10 + total HD of the enemies.
If the roll fails after the characters brancish their weapons, the fight will begin and everyone has to resort to violence.\par

Alternatively, the characters may try to ready their weapons secretly.
Have them roll Sleight of Hand (Dex) against the closest enemy.
If they pass, they gain an advantage on the Initiative roll.
If they fail, they still gain the advantage, but the enemies will also ready their weapons and they will also gain the same advantage.\par

% One alternative is to try to de-escalate the situation.
% TODO: Persuasion rolls, but like the death saving throws. Need three winds or three losses. For added drama, raise the stakes from the battle at each turn...
% Persuasion rolls are at a disadvantage if they brandish their weapons.

% TODO: Intimidation while the battle continues.

\subsection{Hauntings}
\subsection{Exorcism}
% TODO: Spell attack rolls, looking for three consecutive wins before getting exhausted... (Alternatively, use HPs) Only Wis based casters can attempt this.
% TODO: Add modifiers based on preparation.
\subsection{Subterfuge}

% TODO: Add rule for false appearance: The difficulty to spot is DC 20 + the monster's CR or the spellcaster's spellcasting abilty.

\section{Conflict Against Natural Forces}
\subsection{Firefighting \& Spread Mechanics}
% TODO: Fire as a monster.
\subsection{Floods}

\subsection{Climbing}
% TODO: Inpiration from the scenes in Damsel
% https://rpg.stackexchange.com/questions/103859/how-do-i-design-a-climb-up-a-cliff-challenge

% TODO: Full Contact Tree Climbing https://www.reddit.com/r/DnD/comments/f33095/what_minigames_do_you_use/



\subsection{River Crossing}
% TODO: Wagon River Crossing https://www.reddit.com/r/DnD/comments/f33095/what_minigames_do_you_use/


\subsection{Navigation}

% TODO: Basic idea: roll survival to offset penalties to speed, or roll survival to open up other areas on the map.

\section{Conflict Against Society}
\subsection{Homelessness}
\subsection{Cities as Monsters}
% TODO: Write the stat blocks as a gargantuan swarm.
\subsection{Governments as Monsters}



\section{Conflict Against the Supernatural}
\subsection{Conflict Against the Fey}
% Refer to https://the-eye.eu/public/Books/rpg.rem.uz/Dungeons%20%26%20Dragons/3rd%20Party/5th%20Edition/Legendary%20Games/Faerie%20Passions.pdf

\subsection{Conflict Against Consmic Forces}

\chapter{Internal Conflict}
\section{Anxiety}
% TODO: Write a stat block as a monster, immune to all damage. Casts Fear as a spell, once per day. Calm can help. Heroism should be fixing it, you should also have a saving throw, and the anxiety can have levels as a spellcaster.
% https://www.reddit.com/r/dndnext/comments/bar5iy/how_to_counter_fear_andor_paralyze/
\section{Insecurity}
\section{Curses}
% TODO: How to overcome curses
% TODO: Specific curses. Trigger when a number is rolled, for instance, 13?
\section{Sickness}
\section{Addiction}
\subsection{Alcoholism}

\section{Other Demons}
% TODO: Consider the madness effects on page 80 from Fraternity of Shadows - Ravenloft - Heroes of the Mists
\subsection{Obsession \& Paranoia}
% TODO: To roll-play obsession, have the player roll investigation checks on the objects of obsession. If they fail, tell them that they do not find anything.
% TODO: To roll-play paranoia, have the player roll perception checks. If they fail, tell them that nobody is following them. Alternatively, tell ask for insight checks in social scenarios, and on failures, tell them that there are no plots against them.


\begin{emphasisParagraph}
	Fantasy demons can become meaningful to players if they subtly symbolize real life demons.
\end{emphasisParagraph}

This is a simple mechanic: As a GM, generate an NPC demon.
Give the demon an agenda, and keep it hidden from the player.
This demon will occasionally communicate with the player.
In fact, it's better if the player summons a demon and decides to enlist their help.\par

If an action that the player is taking is contributing to the demon's agenda, the player will get a bonus die of $d4$, $d6$, $d8$, $d10$ or $d12$.
(Get the demon's hit dice, round up to the nearest die.)\par
As the GM, you decide whether the player gets the additional die. Make sure that the player knows the bonus die, and in fact, make them roll it.
This die also stacks with \emph{Bless} and Bardic Inspiration.\par

The purpose of this mechanic is to build tension and suspense. It should be occasional, and over time in a longer campaign, the player will finally triangulate what the demon's agenda is.\par

\subsection{Optional Rule: Demon Gets Stronger}
Track every time a Nat 20 is rolled on the rolls where the demon bonus applied.
This is the number of times that the demon can cast \emph{Hold Person} at will on the said player.
Cue the player either before or after the first nat 20 roll.
This is important to let them know.
Once the player is aware that this is giving the demon some sort of control over them, they can choose to not apply the bonus.

Then put them in desperate situations where they need the bonus.\par


% TODO: Add a list of possible agendas, or refer to Fishel 2023.

\subsection{Playtest Notes}

% TODO: Add


\chapter{Spellcasting}
\section{Learning Spells}

\subsection{Conjuration Spells}

% TODO: The player can learn spells by defeating elementals and aberrations. Give a template for each level, explain the filter and restriction.
% TODO: Optional rule: Add a check to control the conjured creature, if the total level is higher than the caster. Or, add the levels of the creatures and roll against that level. Or have each creture roll a Wisdom saving throw against the caster.
% Cite the ``Kobold Guide to Magic''



\subsection{Playtest Notes}

\section{Player Creativity in Spellcasting}

\begin{emphasisParagraph}
	Use this if you want to allow for some creativity in spellcasting (without breaking the balance).
\end{emphasisParagraph}

\begin{emphasisParagraph}
	How many times have players asked you whether they can light up the grease from the \emph{Grease} spell?
	Typically, the DM has to say no. This section suggests a way for allowing for player creativity while keeping it balanced.
\end{emphasisParagraph}

D\&D uses the ``Vancian'' spellcasting paradigm: there is a known recipe with a known effect.
If you follow the recipe, you get the effect.
There is no written way to get a slightly different effect in the rules.\par

Other systems, such as Mage the Ascension in the Chronicles of Darkness, allow for limitless creativity.
This is fun, but can make things difficult for the GM.\par

The rules in this chapter are for getting the best of both worlds.
As a GM, you do not refuse the creative players, but you do make things challenging, so that you do not lose control over the texture of reality.

\subsection{``Nudge'' the Spell}

The spellcaster wants to change one aspect of the spell, without changing the damage, level, or duration.
For example, the damage type may change.
The area of an area effect spell might change.
Alternatively, what the spell does can change.
This should not be a change that gives an absolute advantage in mechanical terms.\par

In this case, the player rolls Intelligence (Arcana) DC 20 + the level of the spell.
If they fail, the spell fizzles with no effect.
If they roll a nat 1, the spell ``backfires'' in some way.
Optionally, if they roll a nat 20, they can add the spell as a new spell.

% TODO: A wonderful example comes from ``Dimension 20'' with the web spell. Add here.

\subsection{Metamagic: Make the Spell More Powerful}

\subsection{Permanency}

% TODO: Why this rule?
% TODO: Mention the rules for permanency in PHB.

Repeat the spell, at the same location. With each repetition, roll an Intelligence (Arcana) check against DC 20 + the spell's level.
\begin{enumerate}
	\item If they roll as many successes as the spell's level in a row, the effect becomes permanent.
	\item If they roll a nat 1, they reset.
	\item If they cast another spell, they reset.
\end{enumerate}

The player can rest in between to replenish their spells.

% TODO: Calculate the odds, we probably want a slightly different rule. This also does not work well with cantrips.

% TODO: Use this to give conjured demons personality and turn them into servants...
% Cite the ``Kobold Guide to Magic''


% TODO: Increasing the impact or area of a spell
% TODO: Joint spellcasting
% TODO: Adding stats to creations
% TODO: Not using components
% TODO: Wild magic backlashes per spell



% TODO: _Awakened_ animals as races
% TODO: _Awakened_ modern objects as races
% TODO: Roleplaying childhood and teenage years

\chapter{Rewarding Creativity}

% Good but futile idea: 25
% Good idea that moved the party forward but created troubles: 50
% Great idea that worked: 100
% Tactical use of spells: divide XP of the monster by the spell level.
% Successful deescalation: XP level
% Attempt at deescalation that failed: XP level / 4
% Party-level achivement: 25 / 50 / 100
% City/County-level achivement: 250 / 500 / 1000
% Country/Plane-level achivement: 2500 / 5000 / 10000
% Continent/Planet/Plane-level achivement4: 25000 / 50000 / 100000


% Reward for achieving goals?

% \chapter{Crafting}
% \section{Alchemy}
% TODO: Potions, Poisons and Bombs
% TODO: Take inspiration from City & Wild, Alchemy section. Simplify and convert to Eberron, Forgetten Realms or other settings. Each plane represents one damage type.

% TODO: Let's follow the cocktail approach: three elements: base, an "essence" based on the planar influence and something else...

% \section{Metalworking: Weapons /& Armor}

% \section{Enchanting}


\chapter*{Suggested Reading}

% TODO: Fix the format

\cite{Fishel2023} % Proactive Roleplaying
% Eberron Linguistics


\printbibliography[heading=none]

\end{document}
