\documentclass[twocolumn]{dndbook}

% Compile using UTF-8
\usepackage[utf8]{inputenc}

\usepackage{xkcdcolors}

\colorlet{dirtyorange}{xkcdDirtyOrange}

\usepackage{mdframed}

\usepackage{hyperref}

\hypersetup{
	colorlinks=true, % Whether to color links (a thin box is output around links if this is false)
	%hidelinks, % Hide the default boxes around links
	urlcolor=dirtyorange, % Color for \url and \href links
	linkcolor=black, % Color for \ref/\nameref links
	citecolor=dirtyorange, % Color for reference citations like \cite{}
	hyperindex=true, % Adds links from the page numbers in the index to the relevant page
	linktoc=all, % Link from section names and page numbers in the table of contents
}

\newenvironment{emphasisParagraph}{
	\begin{quote}
	\begin{mdframed}[
		topline=false,
		bottomline=false,
		rightline=false,
		skipabove=\topsep,
		skipbelow=\topsep,
		linecolor=xkcdBottleGreen,
		linewidth=3pt,
	]
	\em
}{
	\end{mdframed}
	\end{quote}
}

\usepackage{titling}
\newcommand{\subtitle}[1]{%
  \posttitle{%
    \par\end{center}
    \begin{center}\large#1\end{center}
    \vskip0.5em}%
}

\usepackage{biblatex}
\addbibresource{ExtensionsD20.bib}



\begin{document}

\title{The DM Guide to Doing Things Differently}
\subtitle{(or The Book of Diverse Challenges)}

\maketitle

% The DM's Guide to Doing Something Different
% The DM's Guuide to Designing Non-Violent Challenges
% The DM's Guuide to Designing Diverse Challenges

\chapter{Introduction}

This book offers mechanisms for a diverse set of challenges.
I wrote this book, because I did not want my games to be stuck in a repetitive pattern.
The traditional game follows the following pattern: three easy combat encouters, party is statistically guaranteed to survive unless they active try to get themselves killed.
Two traps. Dungeon crawl. Boss battle, a few of the player characters may die.
Characters met in a tavern, don't ask why they were in the tavern or why they decided to walk out together.
Players and the ``bad guys'' are both trying to kill each other, but we do not know why exactly.\par

Even the pulpest pulp action movie has mor challenges.
Action movies will have conflicts with nature,

\begin{emphasisParagraph}
	Fantasy literature has moved beyond the trope of a band of people going and killing a boss.
	Role playing game mechanics have not, and it's getting boring.
	How to make it more fun?
\end{emphasisParagraph}


Fantasy literature itself has long moved beyond this formula.
We do not watch Game of Thrones or Witcher just for the boss battles --- combat encounters are only a small portion of the entire set of challenges.
Up until the boss battle, we watch the characters deal with social drama, political intrigue, conflicts with nature and internal conflict.
Seeing characters in these diverse set of encounters makes us relate to them, and care about them.
I wanted to capture all of this in my games, so I started looking for mechanics for a diverse set of challenges.\par

The traditional challenges in fantasy games, are combat, typically to death, and traps and puzzles.
Typically, games seem to be stuck in a false dichotomy of roleplay vs combat.
However, fantasy captures readers when they can relate to the characters in some way.
To that end, this book strives to provide ways to create encounters to make our characters relatable.\par

\begin{emphasisParagraph}
This book adds mechanics for a diverse set of non-combat challenges and an option to bulid diverse \& relatable characters.
\end{emphasisParagraph}

Most non-combat challenges in roleplaying games are resolved through either roleplay, or just one skill check.
Roleplay is fun. However, I find that roleplay can be burdensome sometimes.
Lots of players simply want to make choices, roll dice based on those choices, develop tactics to win.
I find that mechanics, player empowerment, and the risk of failure create enjoyment for many players.
This means that we involve the dice to simulate stakes.
Therefore, the guiding principle is that if there is going to be roleplay, it should be accompanied by roll-play.\par

\begin{emphasisParagraph}
	Guiding principle 1: if there is going to be a challenge resolved through roleplay, combine this with roll-play mechanics.
	The possibility of failing makes it more fun and informs the roleplay.
\end{emphasisParagraph}

At the beginning of each chapter, you can find why you might want to use the mechanics in the chapter.
If those are not of interest to you, feel free to pass them by.
At the end of each chapter, are playtest notes.
This keeps me honest --- do the rules and mechanics work as intended?
They may also be useful for inspiration.\par

\begin{emphasisParagraph}
	Guiding principle 2: Give players multiple paths for resolution.
\end{emphasisParagraph}

One difficult that I personally had with my first D\&D campaign was how NPCs were able to arbitrarily ``do stuff''.
Necromancers will create armies of undead, but my necromancer PC cannot.
There may be flying cities in the setting, but the best I can do with a spellcaster is the \emph{fly} spell.



\begin{emphasisParagraph}
	Guiding principle 3: If an NPC can accomplish \emph{X}, there should be a way for PCs to accomplish \emph{X} as well.
\end{emphasisParagraph}

Perhaps another way to look at D\&D is: it is collective storytelling and
role-play that is interspersed with one type of minigame: the combat encounter.
This book aims to diversify and increase the types of minigames.

Feel free to use each chapter separately, or altogether. Enjoy.




\chapter{d20 Real Combat}
\label{chap:d20_real_combat}

\begin{emphasisParagraph}
	\paragraph*{Consider using this mechanic if} any of the following is true:
	\begin{enumerate}
		\item you want to run a survival themed session where party members may die
		\item you want to add realism to combat to get your players to strategize and plan ahead of combat
		\item you want to dissuade your players from jumping to combat-based solutions
	\end{enumerate}
\end{emphasisParagraph}

% TODO: First edit

\subsection{Hit Points Break Immersion}

Your player overcomes multiple challenges through creative solutions and makes the rolls.
They are in the bedroom of the local tyrant. They want to slit their throat and kill them.
They can make an roll with advantage, pass all checks, but they are still unlikely to dispatch the enemy with one blow.
It is almost as if all of that protection was for nothing - these ``boss'' NPCs are almost impervious to anything.
However, if you repeatedly hit them with the pointy end, eventually they will succomb and fall.\par

My point is that this breaks anyone's immersion.
Computer games allow for stealthy play --- we should be able to do so in a game.
However, the hit point mechanic is one of the oldest and most established rules.
There are multiple reasons why they stood the test of time --- almost anything else is too difficult to play.
Instead, we will make a compromise.
We will use the hit point mechanic, but make a few changes so that violence is realistic.

\subsection{d20 Real Combat Rules}

Make three changes to the usual hit point mechanic:
\begin{itemize}
    \item No level bonuses to hit points.
    \item Hit points are regained fully at the end of each round.
    \item Potion-drinking is a bonus action
\end{itemize}

\paragraph*{No level bonus to hit points}
The first amendment means that humans, humanoids and all players medium sized races have only one hit die worth of points, based on their class and constitution.
Larger and magical creates still have higher hit points, and will be extremely difficult to kill.

\paragraph*{All hit points are regained fully at end of each round}
The second rule means that unless an attack or multiple attacks reduce a character to 0 hit points, they do not die.
This makes it easy to kill player characters, but really difficult to kill large or gargantuan monsters who have lots of hit points thanks to their size.\par

The purpose of these changes is to to incentivize players to avoid combat if possible, and challenge them to find other solutions to problems.
If the combat does take place, we incentive them to think it through and strategize carefully.
Once it does take place, it will feel epic, because each round is going to be high stakes.
Alternatively, it will feel satisfactory to players, because it will be an very well-planned ambush where the enemy does not even get a chance to fight back.
If the plan fails, or if the players themselves are ambushed, they will be compelled to use potions, powers and inspiration points to save themselves.
This means that a high-level fighter can easily die from a goblin ambush, and a dragon is so difficult to kill that requires either a group of characters that can deal very large amounts of damage, or an archer batallion.\par

If an attack hits, but does not deal enough damage combined with other attacks in the same round to bring the HP to 0, this means that the target survives.
You may narrate this as dodging, or a hit being deflected by armor, or a scratch or a bruise.
Alternatively, if the characters drink a potion, it may mean that they were wounded, but the wounds healed magically.
The final interpretation is up to the GM.
When the players drop down to 0 HP, this means that they are mortally wounded, and the death saves start as usual.

\paragraph*{Potion drinking is a bonus action} To give players more options, we make potion-drinking into a bonus action.
Unless the potions are consumed within the round, they lose their usefulness under these changes.
If the characters have their potions on a belt or in an accessible location, they can drink the potion as a bonus action.
However, taking a potion out of a backpack is a full action.\par

\subsection{Examples}

\paragraph*{Example: Ambushing spellcasters} This mechanic makes battles with mages deadly.
If a party attacks a spellcaster with no plan in mind, the first PC that the mage casts \emph{Magic Missile} on is almost certain to lose consciousness.
This adds realism and high stakes, it is similar to a party attacking a human with a modern automatic handgun.
It is also in line with most of fantasy literature as sellcasters become very powerful adversaries.
What can a party do? They may try to obtain protection, and then trick the spellcaster into attacking the party member with the protection, while the rest of the party tries to ambush them within one round.
Alternatively, healers may simply hold their action to heal the hurt party members.
They can distribute potions to all party members, have them ready and take them as bonus actions.
Finally, they can decide to sneak up to the spellcaster to prevent them having a chance.
One way or the other, the spellcaster needs to be dealt with in one or two rounds, otherwise the party risks losing a character.
This high-stakes and fast combat is more reminiscent of real-life combat, and is intended to increase both the climactic tension, and increase the fun through planning for the climactic battle.\par

\paragraph*{Example: Ambush on a Troll}
The party spots a troll resting by a campfire. The wizard opens with \emph{burning hands},
catching the troll off guard and setting its wounds alight. As the flames lick at the troll,
the fighter charges in, landing a solid blow with their sword. At the same moment, the rogue
emerges from the shadows and lands a devastating backstab. The combined assault, especially the
fire damage, overwhelms the troll before it can react—demonstrating how teamwork and exploiting
weaknesses can bring down even a fearsome foe in a single, well-coordinated round.\par

\paragraph*{Example: Troll Gains the Initiative}
In a different encounter, the same troll surprises the party, bursting from the
underbrush with a guttural roar. It wins initiative and immediately attacks,
raking the nearest character with its claws. The party scrambles to respond,
but the troll’s ferocity and regeneration make it a terrifying opponent when
it gets the drop on them. The players are forced onto the defensive, and without the
advantage of surprise or preparation, the fight becomes a desperate struggle for
survival—highlighting how initiative and tactics can dramatically change the outcome of combat.\par


\paragraph*{d20 Real Combat: How to have fun}
If the party can (1) spot the troll before it spots them, (2) coordinate their actions with
a bit of preparation and discussion, this encounter can be over in a single round.
However, if the player rush in, and the \emph{burning hands} is cast in one
round and the backstab happens in another round, the troll becomes a very fearsome
foe, and the encounter may even end in a TPK.\par

One way to onboard players is to give them a heads-up that they need to be careful
and that they need coordinate with each other. If they fail to do so, the troll
may act on instinct, and not follow them if they start runnign away. As the tables turn against them,
let them know that they can run away while there is still time. Next time, they
will coordinate with each other.\par

Overall, to have the players have a good time, you want them to suffer a bit,
learn from the suffering and mistakes, and let them try something else, and then
win. This kind of ``real'' combat is a good way to get bored players a new
way to play the game.\par





\subsection{Playtest Notes}

I have not playtested this part at all! If you use them, please contact me to add, I would appreciate that a lot!




\subsection{Future Work}
I am planning to add mechanics for being bruised or wounded.


% TODO: Add a condition called "Bruised" where you get 1d4 to intimation, substract 1d4 from all other Charisma checks.
% TODO: Add a condition called "Wounded" where you substract 1d4 from all "physical" checks.



\chapter{Specializations: d20 + d4}
\label{chap:d20_plus_d4}

\begin{emphasisParagraph}
	\paragraph*{Consider using this mechanic if} you want to
	\begin{enumerate}
    \item reward your players for writing detailed backstories
    \item add some character quirks without breaking the balance of the game
    \item allow creation of classes characters (diplomats, sages, merchants, or even commoners) while keeping them relevant and sufficiently powerful.
    \end{enumerate}
\end{emphasisParagraph}

\paragraph*{TL;DR} Add a \emph{specialization die}, like the Bardic inspiration or \emph{Bless} to the rolls.


SRD 5e is great in its simplicity, but makes it difficult to customize characters with specializations.
Consider checks to find out if a character knows about the climate of a country that they did not visit,
or international politics, or the main trade routes in a region.
The DM will likely default to Intelligence (History) checks, even though these are not related to history.
In contrast, previous editions of SRD would have had specific knowledge checks, but then these were rarely useful,
so it was not rewarding to invest in these at character creation.\par

To get the best of both worlds --- in depth character creation and simplicity --- we introduce ``specializations''.
If, a character spends more than 8 hours a day for at least 5 years in a profession, on a hobby or an area of study,
they get the ``specialization''.
This gives them a $1d4$ die to proficiency checks, stacking with \emph{Bless} and Bardic inspiration.\par

\paragraph*{Preventing Critical Failures}
Unlike \emph{Bless} and Bardic inspiration, this die can prevent a critical failure:
So long as the specialization die is not $1$, the DM can present the situation as a ``close save''.
The roll can still be a failure, though, if the total does not pass the DC.\par


\paragraph*{Coherent Backstory} Players who want to have these specializations,
need to account for all of the years in their backstory.
For example, if a character spent at least five years working as a tailor prior to adventuring,
they can get the ``tailoring'' expertise.
They need to spend at least 8 hours a day to be a specialist in these area, meaning that they have little time for anything else.
There is no way that they apprenticed with a carpenter or a blacksmith since they already
had a day job, so they cannot get these specializations from the same years.
This also means that they are not a noble, and they are a character who had to earn money to make a living\par

To make things easier, we can add specializations to a character \emph{as they become relevant}.
Unlike the usual proficiencies, specializations will come up less frequently, and can be developed
retrospectively.\par

\paragraph*{Flexibility, balance and fun}
Suppose that the previous ``tailor'' character's player also wants them to have a specialization in ``Animal Handling --- Horse''.
The player might claim that the character spend time riding and taking care of their horse.
This is probably a difficult thing to achieve: if they were a tailor, they were probably in the city, how did they find a place to keep the horse?
How did they pay for a horse --- owning and maintaining a horse is a luxury for a city-dweller.
How did they even find the time and energy?
One way that the DM may go is to rule that this is inconsistent with the backstory and refuse.
However, another way to deal with this is to direct these questions to the player, and watch a more colorful character grow.
Maybe, they rode their horse beginning all the way at age 15, for over ten years, at least a few hours every night.
(So that the total hours match.)
Maybe they had to move to the city and leave their beloved horse behind\ldots
It can come together to make a coherent whole.
The whole purpose of ``specialization'' is to make character building fun and go beyond the usual fantasy roles.\par


\paragraph*{Long-lived races} This rule intentionally gives long-lived races an advantage.
This is in parallel with fantasy literature and common sense: a hundred year old elf is
bound to have some advantages over their 20-year-old human companions, although perhaps not in adventuring.\par

\begin{DndTable}[header=Specializations]{X}
	\\	Carpentry
	\\	Massage
	\\	Roof Tiling
	\\	Shoemaking
	\\	Sailing
	\\	Baking
	\\	Boat Building
	\\ 	Brewing
	\\	Winemaking
	\\	Firefighting Techniques
	\\	Stonemasonry
	\\	Swimming
	\\	Animal Handling --- Specific Animal Species
	\\	Singing
	\\	Comedy / Joking
	\\	Musical Instrument --- Specific Instrument
	\\	Embroidery
	\\	Knitting
	\\	Painting
	\\	Drawing
	\\	Cooking
	\\	Algebra
	\\	Anatomy --- Humanoids
	\\	Anatomy --- Aberrations, Large
	\\	Anatomy --- Aberrations, Small
	\\	Anatomy --- Aberrations, Medium
	\\	Anatomy --- Aberrations, Huge
	\\	Anatomy --- Beasts, Large
	\\	Anatomy --- Beasts, Small
	\\	Anatomy --- Beasts, Medium
	\\	Anatomy --- Dragons
	\\	Anatomy --- Fey, Tiny
	\\	Anatomy --- Fey, Small
	\\	Anatomy --- Fey, Medium
	\\	Anatomy --- Fey, Large
	\\	Anatomy --- Fiends, Large
	\\	Anatomy --- Yugoloth
	\\	Anatomy --- Monstrosities, Large
	\\	Anatomy --- Monstrosities, Huge
	\\	Anatomy --- Monstrosities, Medium
	\\	Anatomy --- Giants
	\\	Astronomy --- Earth
	\\	Astronomy --- Faerun
	\\	Astronomy --- Eberron
	\\	Astronomy --- Oerth
	\\	Bookkeeping / Accounting
	\\	Botany / Herbes
	\\	Alchemy / Chemistry
	\\	Philosophy
	% \item Politics
	\\	Theology --- Specific religion
\end{DndTable}



\paragraph{*Adding new specializations} Here is the rule of thumb:
Specializations have to be one of the following:
\begin{enumerate}
    \item something learned through an apprenticeship or receive on-the-job-training
    \item something learned through a 100-level university course, or an equivalent basic course in the campaign setting
    \item something learned or developed through night classes
\end{enumerate}
``Specialization'' has to be more specific than the usual proficiencies in the fifth edition,
that is the point. However, if you go more specific than a 100-level university course, then
it becomes too specific to be relevant for most game sessions. This rule of thumb
ensures that you stay in the

\paragraph*{Increasing the specialization die}
Specializations get better only through spending time on them.
Use the following table to increase the die.
\begin{DndTable}[header=Time Spent \& Die]{XX}
Time Spent	&	Specialization Die \\
5 years		&	1d4 \\
10 years	&	1d6 \\
50 years	&	1d8 \\
100 years	&	1d10 \\
500 years	&	1d12 \\
\end{DndTable}

This reflects the diminishing returns to specializing in one area.
At the same time, some non-human races or ancestries can be used to create in depth characters.
This can also be used to make some weak NPCs really powerful in one specific area,
for example, you could have an elven botanist who spent an entire lifetime with plants, and is excellent at brewing potions.\par

\paragraph*{Examples with Extra Rolls}
Suppose a character has a specialization in ``Anatomy --- Trolls'' (1d6) and is trying to identify a troll’s weak spot during combat.
They roll their normal Intelligence (Nature) check, add proficiency if applicable, and then roll an extra 1d6 for their specialization.
If the result is high, they might recall that fire or acid is needed to stop a troll’s regeneration, giving the party a crucial tactical edge.\par


A bard with ``Singing'' (1d4) attempts to impress a noble at court.
They roll Performance (Charisma) as usual, but also add 1d4 from their specialization.
This bonus could be the difference between a polite applause and a standing ovation (or a new patron).\par

A gnome wizard who studied Alchemy all his life before starting adventuring, has a specialization in ``Alchemy / Chemistry'' ($1d8$).
When they are brewing a rare potion, on their Arcana check, they add their $1d8$ specialization die, reflecting years of focused study and practice beyond what most wizards know.\par

\paragraph*{Real-life Specialization Examples}
A librarian (specialization: ``Library Science'' $1d6$) is searching for a lost tome in a vast archive.
When making an Investigation (Intelligence) check, they add their $1d6$ specialization die, representing their deep familiarity with cataloging systems and obscure references.\par

A merchant (specialization: ``Trade and Negotiation'' $1d8$) is haggling over the price of rare spices.
On a Persuasion (Charisma) check, they add $1d8$, reflecting years of experience reading people and knowing when to push for a better deal.\par

A fashion designer or socialite (specialization: ``Fashion and Trends'' $1d4$ or $1d6$) is attending a royal ball.
When making a Deception (Charisma) or Performance (Charisma) check to impress or blend in, they add their specialization die, showing their expertise in style, etiquette, and reading the room.\par


\section{Optional Rule: Local Specialization}

A character, say, a ranger, with proficiency in Survival will get the proficiency bonus anywhere.
However, it is reasonable that they get a bonus if they have been in the same area for years.
Similarly, an investigator will roll Investigation (Int) anywhere, but if they are investigating in the neighbourhood that they lived for the past several years, they can get a bonus.
Use the specialization die from the table: for example, if they lived in the same area for the last 5 years, they get a $1d4$ specialization die.\par

The purpose of the mechanic is to make the backgrounds of the players meaningful in terms of mechanics.
This can help players at lower levels in a believeable fashion. As they level up, the player can share the stress their character getting outside their local district as they lose the die.\par

For the purposes of this mechanic, consider the smallest area where everyone will likely be at most one degree of separation from you.
This can be a 100k population city, a small town and surrounding hamlets, a county, or a district up to 100k people in a large city.\par

\section{Optional Rule: Age \& Wisdom}

In the first season of the Netflix show Arcane, (based on League of Legends) Professor Heimerdinger points out that he is 307 years old, and has more life experience than the rest of the councillors.
In Lord of the Rings, Elrond the Half-Elf points out that he has been with Isildur when he failed to cast the ring into the fires of Mount Doom. He is respected for knowledge of history and understand, and is revered as a wise figure.
In D\&D, you easily get elves and half-elves that lived much longer than the rest of the party, and who have little or nothing to show for their Wisdom.
This mechanic is meant to capture the reality of living a long existence and the wisdom and insight that comes with such long life experience.
As a mechanic, this tips the balance in the favour of elves, gnmomes, and intelligent undead beings.
My method of dealing with this inbalance is to give these characters to players with more experience, or just more mature players and tell them that they have a responsibility to the rest of the party.\par

Simply add the die to all relevant Wisdom and Insight checks.
This does not include the spellcasting bonus --- the point is to show the wisdom that comes with age.
However, at DM's discretion, it can include Wisdom saves.\par

\begin{DndTable}[header=Life Experience Bonus]{XX}
	Age			&	Specialization Die \\
	35+ years	&	1d4 \\
	70+ years	&	1d6 \\
	100+ years	&	1d8 \\
	200+ years	&	1d10 \\
	300+ years	&	1d12 \\
	500+ years	&	1d20 \\
	\end{DndTable}


Because this is an inbalanced mechanic, it needs to be implemented with discretion.
It is up to the DM to give these characters to more mature players in the party, or simply not use the mechanic or involve it only in certain situations.\par

\subsection{Example}

An old commoner can have this bonus, since they are advanced in their years.\par

Players can enjoy the privilege of coming from one of the longer-lived races, such as dwarves or elves.\par

\section{Optional Rule: Forbidden Knowledge \& Power}

% TODO: The essence of the ``forbidden bonus'' is this: Give them a bonus die, but tell them that a $1$ from the bonus die is also a Nat 1. For dice The idea is this: they freely get this extra boost, and it is easy to obtain, but something nefarious is likelier to happen.
Almost all cultures have an age level where an individual is expected to be ``wise''.
Consider giving an advantage to all Wisdom rolls over a certain age, both to players and to NPCs.
In my games, I consider this age to be 60.
This will give elven and even dwarven characters a serious advantage (pun intentional).
However, it is a more realistic depiction of these characters, and is a way to recapture the position of elves and dwarves in relation to humans.

\chapter{Dramatic Mechanics}

\begin{emphasisParagraph}
	\paragraph*{Consider using these mechanics if} you want to give challenges other than combat to your players.
\end{emphasisParagraph}

There are a host of dramatic challenges in literature and movies.
However, the two climatic challenges in role playing games are puzzles and combat challenges, typically fought to death.
Whenever there is another challenge, this is resolved through roleplay, and typically, one proficiency roll.
This chapter aims to give combat-like mechanics for other types of challenges.\par

Most of the challenges come in two stages: preparation and the actual challenge.
This is intended to increase tension leading up to a climactic resolution.\par

\section{Exams}

It is uncommon to have exams in fantasy settings, but being student, or needing to pass exams is a common experience that almost all players will relate to.\par

There are multiple reasons to have an exam as part of an adventure. Here are some:
\begin{enumerate}
    \item exams may be a good alternative to the ``meet in a tavern trope''
    \item an exam may be required to be members of an organization
    \item adventurers may be required to pass an exam to gain access to institutions of learning
\end{enumerate}


\paragraph*{Preparation}

Here is the basic rule for studying for an exam:
Step 1: Roll Insight (Wisdom) with a DC of 11.
Step 2: If you pass, you get a ``preparation bonus'' of +1. However, it does not end there. You can now roll against a difficulty of DC 12. If you pass this one, your ``preparation bonus'' increases to +2.
Step 3: Repeat the roll with incrementing DC until you fail. For example, if you failed at DC 15, your ``preparation bonus'' will be +4.

In case of a critical failure, decrement the bonus by 1. In case of critical successful, get advantage on the next roll.
Each roll takes 1 hour of time in the game, and exhaustion rules may apply. You can prepare for an exam only during the week leading up to the exam.\par

If the player did not go to any one of the classes, all rolls are at a disadvantage. If the player skipped some classes, start at the 4th roll and roll disadvantage afterwards.\par

Preparing cheat sheets and cheating apparatus is another alternative.
Ask the player what exactly they are doing.
Depending on the answer, get them to roll either a proficiency check with a Forgery Kit, Calligraphy Kit, or a perhaps Cartographer's Kit (Intelligence).\par

Alternatively, Artificer characters may prepare infusions, or spellcasters may prepare spells, or all classes may attempt to hide an item with an Intelligence check.
A disguised \emph{Helm of Telepathy} or a prepared \emph{Detect Thoughts} can be very useful.\par

% TODO: Forgery Kit is from Xanathar's Guide to Everything. Helm of Telepath is DM's guide.

Things can happen during preparation. If desired, here are some ideas and a table.
\begin{DndTable}[header=Random Event While Preparing for an Exam]{rX}
	Roll	&	Event \\
	1		&	A friend invites for a night out \\
	2		&	A rival comes over and brags about what they did for study. \\
	3		&	Exam questions are rumoured to be available. Pay to get the questions, or continue studying? If real, going over the questions will give an adventage on the day of the exam. \\
	4		&	Player gets a moment of epiphany. Advantage on the next study roll.
\end{DndTable}


\paragraph*{Day of the Exam}

An exam will have three or four questions or sections. Roll Investigation (Intelligence) against DC 15 for each question, and add the ``preparation bonus'' from earlier.\par

If the characters are trying to smuggle in cheating equipment, such as magic items or cheat sheets, succesfully hiding them would require a Deception (Charisma) or a Sleight of Hand (Dexterity) check.\par
Use the following as a guide to determine the DC:\par
\begin{DndTable}[header=Smuggling Cheat Sheets into the Exam]{Xl}
	Exam	&	DC \\
	No search, just a midterm with little oversight	&	5 \\
	Midterm with some tired assistants	&	10 \\
	Finals with multiple vigilant assistants	&	15 \\
	High-stakes exam & 20 \\
\end{DndTable}

If desired, you can roll a ``luck'' $d20$ and determine the result as the DC. Perhaps the proctors decided to be extra vigilant on that day\ldots\par

Cheating will typically require a Sleight of Hand (Dexterity) check, typically against the passive perception of the nearest proctor. (Use the DC from table above as the passive perception, or roll a $d20$ to determine. If they are unlucky, the proctor will be particularly vigilant.)I
If copying from another person, roll Investigation (Intelligence) against DC 15 to see \emph{if they got it right}.\par

% TODO: Link to Eberron Papers Please

\paragraph*{Aftermath}

Consider giving a bonus if the players pass a high-stakes exam without cheating --- see \nameref{chap:d20_plus_d4}.

\subsection{Playtest Notes}

I used exam preparation and exam mechanics for a session 0. The players met each other during a course. I gave them an exam subject, and asked them why they were taking the exam. This cued them to create characters with more detailed backgrounds.
I also clearly gave them the option to study or to cheat, which made it more fun by giving them a choice. The study mechanic was a bit too complex and got in the way of enjoyment while explaining.\par


\section{The Language Gap}

\begin{emphasisParagraph}
	In real life, there is no language called ``Common''.
\end{emphasisParagraph}


``Common'' is a good simplification, but removes a lot of the challenges that can be used in gameplay.
Misunderstandings across languages is a common thematic element and cause for dramatic tension.
The table below provides a mechanic to play out language comprehension.\footnote{The rules here are inspired by the wonderful work \href{https://www.dmsguild.com/product/262786/Languages-of-Eberron}{Languages of Eberron}.
It is pay-what-you-want, please consider using it if you are into linguistics and want to use that in a campaign in Eberron.}\par

\subsection{Language Proficiency}
All languages come with a proficiency level: ``Basic'', ``Intermediate'', ``Advanced'' and ``Native''.

At character creation, choose one or two ``native'' languages in a way that makes sense with the charcater backstory.
These are languages that you either spoke at home, or spoke during your teenage years with your peers.\par

Based on the players handbook, number of languages that a character can speak is a trait based on race.
All characters begin with two languages, except for high elves and half elves, who speak three.
To keep it consistent, give each character 7 points to allocate, or 10 if they are half-elf or high elf.
Allocate the points based on the following table.
\begin{DndTable}[header=Language Proficiency Levels]{Xrr}
	Proficiency Level	&	Point Cost & DC to Communicate\\
	Basic	&	1 & 15\\
	Intermediate	&	2 & 10\\
	Advanced	&	3 & 5\\
	Native & 4 & 0\\
\end{DndTable}

During the game, understanding a ``familiar'' language is a DC 20 Intelligence check.
Each level in the language proficiency drops the DC by 5.
So if you are an ``advanced'' speaker of Elvish, but it is not your native language,
you can roll againt DC 5 to understand what people are saying.
It is only when you reach ``native'' proficiency that the DC drops to 0.\par

Depending on the setting, ``familiar'' may mean either ``same language family'' (like Indo-European languages in our world),
or just the regular languages: elvish, dwarven, halfling and gnome. Languages that are not ``familiar'',
either because they are unfamiliar to the characters, such as Dranonic, or languages that are from other language families add +5 to the DC to understand,
and another +5 if they are using another script, to a grand total of DC 30.
This reflects that it is almost impossible to understand language with a different writing system or speech in a language that shares almost no vocabulary.\par

If using this rule, \emph{Comprehend Languages} adds +20 to the roll to understand.\par

\subsubsection{Optional Rule --- Reading \& Illiterate Characters}
The rules for understanding and communicating in a language are same for written and spoken.
This is to keep it from becoming too complex, but if you like, you can make the roll at an advantage if trying to understand written language, or communicate in written form.
This advantage reflects the ease that comes from being able to see the entire passage at once.
However, if the written language is in another script or form of writing, make the roll at a disadvantage.

\subsubsection{Optional Rule --- Pidgin Languages}
Pidgin language is a grammatically simplified mode of communication between two languages.
If using a pidgin languge in the campaign, add +5 bonus if the character has proficiency at least one of the languages.



\subsection{Playtest Notes}

I playtested this in a campaign that involved goblin communication to two encounters. My players loved it.\par

If there is no combat situation going on, it is best to make one roll at the beginning of a dialogue.
Ask for a second roll once something important comes up, such as the a critical piece of information.
Otherwise, the rolling gets tedious very quickly.





\section{Job Interview}
\label{sec:job_interview}

% TODOL Take out this section entirely? Or simplify and move to the bottom?

Player can be taking the interview, or interviewing an NPC --- this mechanic allows for both.\par

\subsection{Setup}
\begin{enumerate}
  \item \textbf{Required Skill.} The interviewer designates one skill (e.g., proficiency with Artisan’s Tools) that’s relevant to the job.
  \item \textbf{Determine DC.} The interviewer chooses a Difficulty Class (DC). If an NPC conducts the interview, you may run up to three “rounds,” increasing the DC each round to represent tougher questions.
\end{enumerate}

\subsection{Making the Rolls}

\subsubsection{Interviewee plays straight}
\subsubsection{Interviewee attempts deception}
Deception (Cha) vs. DC

% \begin{DndTable}[header=Interview Outcomes]{XrrX}
% \toprule
% \textbf{Situation} & \textbf{Interviewee’s Roll} & \textbf{Interviewer’s Roll (Insight)} & \textbf{Outcome} \\
% \midrule
% Interviewee plays straight
% & Required skill vs.\ DC
% & Insight vs.\ DC
% &
% \begin{itemize}
%   \item Both succeed \(\implies\) Interviewer is convinced of competence.
%   \item Interviewee succeeds but interviewer fails \(\implies\) Interviewer remains uncertain.
%   \item Interviewee fails \(\implies\) Interview fails outright.
% \end{itemize}
% \\[1ex]
% Interviewee attempts deception
% & Deception (Cha) vs.\ DC
% & Insight (Wis) vs.\ DC (contest)
% &
% \begin{itemize}
%   \item Interviewer rolls higher \(\implies\) Deception uncovered; interview fails.
%   \item Interviewer rolls lower \(\implies\) Interviewer believes the lie (treat as straight success if DC was met).
% \end{itemize}
% \\

% \end{DndTable}

\noindent\textbf{Roll Procedure:}
\begin{enumerate}
  \item The player chooses \emph{either} to roll the required skill \emph{or} Deception.
  \item The interviewer always rolls Insight against the same DC.
\end{enumerate}

\subsection{Key Benefits}
\begin{itemize}
  \item \textbf{Player Agency:} The interviewee chooses how to approach the check (honest skill test vs.\ bluff).
  \item \textbf{Meaningful Insight Rolls:} Even when PC–NPC roles are reversed, Insight rolls keep everyone uncertain and engaged.
\end{itemize}

\section{Variant: Reading Intentions}
Use the same approach whenever a character tries to discern NPC motives:
\begin{enumerate}
  \item \textbf{Set a DC} appropriate to how opaque the NPC’s motives are (e.g., DC 12).
  \item \textbf{Player rolls Insight} vs.\ that DC.
  \item \textbf{NPC may roll Deception} vs.\ the same DC as a reaction.
\end{enumerate}
\begin{itemize}
  \item Success without contest: PC thinks they understand the NPC.
  \item NPC’s Deception beats Insight: PC is misled, even if they succeeded on their roll.
\end{itemize}

\section{Playtest Notes}
\begin{itemize}
  \item \textbf{Unplaytested.} Please try this at your table and share any feedback on pacing, balance, and player engagement.
\end{itemize}

\section{Future Work}
At present, no planned revisions—but open to community suggestions!

\section{Running a Business}

\begin{emphasisParagraph}
	Some players want a source of income,
	and many people dream of owning their business.
	The rules here are meant to help DMs guide
	that fantasy in role play.
\end{emphasisParagraph}


Running a business is a ``downtime activity'' in the DM's guide.
The rules involve set costs and a $d100$ table, on page 129 in the 2014 edition.
You also need to use the cost table on page 127.\par

Like most optional rules on the DM's Guide, this is perfectly fine and a great way to run a campaign.
You can also use the rules found on this web page to add a bit more flair:
\url{https://www.thievesguild.cc/core/businesses}\par

If you want something a bit more involved, you can run shops as their own NPC or monster.\par

\begin{emphasisParagraph}
	When you think about it, a business is really an incorporeal monster.
	It is immune to all types of damage, but receives damage
	cash flow. It has a set of stats, and takes actions.
	It can be competed with, and it can go bankrupt.
\end{emphasisParagraph}

\begin{DndMonster}[width=.5\textwidth - 8pt]{Shop}
	\DndMonsterType{Construct (Business), incorporeal, lawful evil}
	% Actions
	\DndMonsterSection{Stats}
	\DndMonsterAction{Daily Cost of Rent (from DMG)}
	\hfill 1 gp
	\DndMonsterAction{Daily Labor Cost (from DMG)}
	\hfill 1 gp
	\DndMonsterAction{Paid-in Capital (analogous to Max HP)}
	\hfill 180 / 180 gp
	\DndMonsterAction{Competitiveness Bonus (analogous to the Attack Bonus)}
	\hfill +2
	\DndMonsterAction{Profit Roll (analogous to the Damage Roll)}
	\hfill $ 1d6 $
	\DndMonsterAction{Proficiency to run}
	\hfill Charisma (Persuasion)

	\DndMonsterSection{Costs}
	Every month, the total monthly cost (30 x total daily cost) is subtracted
	from the paid-in capital, before the profit roll.
	If the paid-in capital goes down to 0, the shop goes bankrupt,
	and there is nothing to recuperate.\par

	The daily cost includes the salary of one skilled worker, as described in the PHB.
	The cost does not include the rent.
	Everything here is written with the assumption that the player owns the place.
	However, if the player rents the place, add the rent to the daily cost.
	Both the value of the property and the rent are left to the DM to decide.\par

	% Actions
	\DndMonsterSection{Actions}
	The shop takes one action every month.

	\DndMonsterAction{Make Profit}
	This is only possible if there are no competitors.
	Roll $d20$ and add the competitiveness bonus against DC 12.
	If you pass, roll for profit, and make profit equal to the roll
	times 30 gp. The gold can be used to replenish the capital,
	and is paid to the shop owner if it is higher than the max capital.

	\DndMonsterAction{Collude}
	This is possible only if there is a competitor, and both agree to collude.
	Roll $d20$ and add the competitiveness bonus against DC 10.
	The competitor also rolls. They share their profits by adding up
	their profit rolls and dividing by 2.

	\DndMonsterAction{Compete}
	This is possible only if there is a competitor. If the competitor
	or the business does not want to collude, they compete.
	Roll $d20$ and add the competitiveness bonus.
	The competitor also rolls.
	Whoever wins rolls for profit, gets twice as many gp as they would have.
	(Multiply the roll outcome by 90 rather than 30.)
	The loser actually loses amount equal to the roll times 30.

	\DndMonsterAction{Replenish Capital}
	As a bonus action, the business owner can increase
	the paid-in capital by paying gold to the shop.
	This cannot go over the max allowed paid-in capital.
	Note that paid-in capital cannot be converted to GP for the owner,
	this is more like inventory or assets, necessary for profit generation.


\end{DndMonster}

\begin{DndSidebar}[float=!b]{Calculations}
% \tocside{Behold the DndSidebar!}
	The profit roll $1d6$ and the DC 12 is chosen for compatibility with
	the ``Running a Business'' rules on page 129 of the DMG 2014.
	They yield very similar expected values, but the distribution is different
	and success depends on the skill of the PC.
	Overall, I wrote the rules keeping consistency in mind with the D\&D economy.
\end{DndSidebar}


\subsection{Cost \& Profit}
According to the DMG, the cost of a shop is based on one employee, whose
salary of 2gp per day. This assumes that there are no rental costs.\par

I also created these rules with the assumption that the player owns the shop.
In the long run, the shop is not expected to be profitable unless
the PC is constantly running the business.
To make a ``shop'' profitable, the player needs to
add ``product lines'' from the section below.
This facilitates plot hooks, and the challenge of keeping the business profitable is part of the fun.\par

If the player wants to be running a business with rent, this will increase the challenge.\par

Finally, it is possible to create businesses with different combinations of profitability and cost.
Use the tables form the subsections ``Different Types of Businesses'' and ``Different Sizes of Businesses'' to create a business
that fits the player's vision. Some combinations are going to be easier to run and will turn into a steady source of income, and others will be more useful for adventures and plot hooks.\par

\subsection{At the Helm: Personally Running the Business}
If the owner of the shop spends majority of the month at the shop or the business,
they can add their relevant skill bonus to the business' competitiveness bonus.
The relevant skill for a merchant is Cha (Persuasion); for a magic shop,
it is Int (Arcana), and for smiths and other trades, it is the relevant tool proficiency.
In other words, the total roll for the month is a proficiency check for the player,
with the additional bonus coming from the shop.\par

Typically, henchmen and skilled workers cannot make this roll. If there is an NPC at the shop,
their salary should be 5gp per day or more, depending on the level of the NPC.

\subsection{Competition}
If the business has no competitors, it just makes profit.
If there is one compettor, the businesses can collude for safety, or compete for profits
and to run the other out of business. The competitor business is an NPC,
and the DM takes the decisions for it.\par

However, if there are multiple competitors, run these as one NPC - see below
for either ``Market'' or a ``Large Market'' as below.
Alternatively, if there is one strong
market player, run this as a ``Company'' - also see below.

\subsection{Product Lines}

A business can hold as many lines as equal to its competitiveness bonus.
Think of these as being analogous to items of a character.\par

To onboard a product line, the shop owner has go through a quest, or
actively search for an opportunity. They then have to pay the
capital requirements.\par

% TODO: Rewrite to simplify.

\subsubsection{Specialty Goods}

Provides a niche advantage in the market.\par

\begin{DndComment}[color=bgtan2018]{}
	\DndMonsterSection{Specialty Goods}
	\DndMonsterType{Product Line}
	\DndMonsterAction{Additional Capital Requirements}
	\hfill 60 gp
	\DndMonsterAction{Competitiveness Bonus}
	\hfill +2
	\DndMonsterAction{Additional Profit Die / Bonus}
	\hfill $1d4$
	\DndMonsterAction{Additional Capital Requirements}
	\hfill 60 gp
\end{DndComment}


Hook: Rumors spread that a guild plans to copy your unique product line.\par


\subsubsection{Luxury Line}

High-margin product line that increases profitability, but risks reduced demand.\par

\begin{DndComment}[color=bgtan2018]{}
	\DndMonsterSection{Luxury Line}
	\DndMonsterType{Product Line}
	\DndMonsterAction{Additional Capital Requirements}
	\hfill 180 gp
	\DndMonsterAction{Competitiveness Bonus}
	\hfill ...
	\DndMonsterAction{Additional Profit Die / Bonus}
	\hfill $1d6$
\end{DndComment}


Hook: Access to rare imports is blocked—can the business secure the supply chain?\par


\subsubsection{Discount Line}

Mass-market strategy that ensures constant sales, but invites competition.\par

\begin{DndComment}[color=bgtan2018]{}
	\DndMonsterSection{Discount Line}
	\DndMonsterType{Product Line}
	\DndMonsterAction{Additional Capital Requirements}
	\hfill 60 gp
	\DndMonsterAction{Competitiveness Bonus}
	\hfill +4
	\DndMonsterAction{Additional Profit Die / Bonus}
	\hfill ...
\end{DndComment}

Hook: Rivals spread rumors that your goods are low quality.\par


\subsubsection{Essential Service}

Resilient product line that guarantees staying power, but limited growth.\par

\begin{DndComment}[color=bgtan2018]{}
	\DndMonsterSection{Essential Service}
	\DndMonsterType{Product Line}
	\DndMonsterAction{Additional Capital Requirements}
	\hfill 60 gp
	\DndMonsterAction{Competitiveness Bonus}
	\hfill ...
	\DndMonsterAction{Additional Profit Die / Bonus}
	\hfill $+1$
\end{DndComment}



Hook: A crisis (drought, war, festival) spikes demand—can the business keep up?\par


\subsubsection{Exclusive Contract}
Guarantees a profit every month. However, if the patron withdraws support,
the additional paid-in capital disappears, and is nor recuperated, and
the additional profit goes away. This can happen at the DM's discretion,
or by rolling a 1 on a $d100$ rolled every month.\par

Note that the capital requirements increase the maximum paid-in capital,
and can be useful for protecting the business.
(Or disastrous if the patron pulls support at the wrong time.)\par


% \begin{DndReadAloud}
%   As you approach this module you get a sense that the blood and tears of many generations went into its making. A warm feeling welcomes you as you type your first words.
% \end{DndReadAloud}

\begin{DndComment}[color=bgtan2018]{}
	% \DndMonsterType{Product Line}
	\DndMonsterSection{Exclusive Contract}
	\DndMonsterType{Product Line}
	\DndMonsterAction{Additional Capital Requirements}
	\hfill 45 gp
	\DndMonsterAction{Competitiveness Bonus}
	\hfill ...
	\DndMonsterAction{Additional Profit Die}
	\hfill $1d4 + 1$
\end{DndComment}

% The |DndSidebar| is not breakable and is best used floated toward a page corner as it is below.

% \begin{DndSidebar}[float=!b]{Behold the DndSidebar!}
% \tocside{Behold the DndSidebar!}
%   The |DndSidebar| is used as a sidebar. It does not break over columns and is best used with a figure environment to float it to one corner of the page where the surrounding text can then flow around it.

%   Use the |\tocside{title}| command to create an entry in the table of contents, if you want.
% \end{DndSidebar}

Hook: The patron may demand favors, errands, or loyalty.\par

The capital requirements and the additional profit die can be set by the DM.
For example, a guaranteed government contract could look like the following:

\begin{DndComment}[color=bgtan2018]{}
	% \DndMonsterType{Product Line}
	\DndMonsterSection{Government Contract}
	\DndMonsterType{Product Line}
	\DndMonsterAction{Additional Capital Requirements}
	\hfill 120 gp
	\DndMonsterAction{Competitiveness Bonus}
	\hfill ...
	\DndMonsterAction{Additional Profit Die}
	\hfill $+2$
\end{DndComment}


This brings in 60 gp a month, guaranteed, but requires a high level of capital.

Hooks: The government may demand favors, errands, or loyalty.
The player may be forced to bribe officials, or navigate bureaucratic red tape.\par

\subsubsection{Original Design}

The player spends some time to create a new product line.
This involves rolling a tool proficiency check, such as Jeweller's Tools, Smith's Tools, or
Alchemist's Supplies, or a relevant proficiency check.

% TODO: Map Cost & Originality to Compatitiveness
% TODO: Map Charisma to Profit Die.

\subsubsection{Illegal Side Hustle}

\begin{DndComment}[color=bgtan2018]{}
	% \DndMonsterType{Product Line}
	\DndMonsterSection{Illegal Side Hustle}
	\DndMonsterType{Product Line}
	\DndMonsterAction{Additional Capital Requirements}
	\hfill 60 gp
	\DndMonsterAction{Competitiveness Bonus}
	\hfill +2
	\DndMonsterAction{Additional Profit Die}
	\hfill $1d6$
\end{DndComment}

Hook: Risk of legal trouble, raids, or moral dilemmas.\par

\subsection{Different Types of Businesses}

The ``Shop'', as described in the DMG and here, is just one type of business.
Here is a set of rules to create other types of businesses.

\begin{emphasisParagraph}
	To create a new business, combine the type of business with the location to determine
	the relevant proficiency, the paid-in capital, the daily cost,
	and the profit die.
\end{emphasisParagraph}

As a rule of thumb, the type of business changes the proficiency roll required when the owner is running the business,
as well as the labour costs and the paid-in capital.
Location changes the profit die and the daily cost of rent.\par

On top of this, the DM can change the daily cost or the paid-in capital for additional storytelling and mini-games.
For instance, preparing three maps for three different shops and
giving all of them a different bonus can be a great way to keep the
players engaged.\par

The DM can also introduce events or hooks related to the business,
such as a rival shop opening nearby, a sudden increase in demand for a product,
a heist targeting the shop's inventory, a supply chain disruption,
a unique opportunity to expand the business with a product line,
a criminal organization trying to extort payments or local officers
trying to collect bribes.\par

Use the following table for the relevant proficiency for different types of businesses.

\begin{DndTable}[header=Business Type]{Xrrr}
	Business Type	&	Proficiency & Cost & Capital \\
	Shop	&	Cha (Persuasion) & 2 gp & 180 gp \\
	Magic Shop	&	Int (Arcana) & 2 gp & 20000 gp \\
	Smithy	&	Str (Smith's Tools) & 2 sp & 240 gp \\
	Bakery	&	Int (Cook's Utensils) & 2 gp & 240 gp \\
	Alchemist	&	Int (Alchemist's Supplies) & 2 sp & 500 gp \\
	Tavern	&	Int (Brewer's Supplies) & 2 gp & 300 gp \\
	Inn & Int (Brewer's Supplies) & 5 gp & 800 gp \\
	Restaurant	&	Int (Cook's Utensils) & 2 gp & 400 gp \\
	Market Stall & Cha (Persuasion) & 0 gp & 30 gp \\
	Jeweller & Dex (Jeweler's Tools) & 2 gp & 1000 gp \\
	Carpenter & Str (Carpenter's Tools) & 2 gp & 120 gp \\
	% Large Market & Cha (Persuasion) \\
	% Large Competitor & Cha (Persuasion) \\
\end{DndTable}

You can also consider having larger and expanded version of each business.
Each skilled worker will require 2gp a day, whereas untrained workers will require 2sp a day.
Hiring experienced experts may cost even more at 2gp.
For instance, a large restaurant may have a head chef (3gp a day),
two sous chefs (2gp a day each), and two kitchen hands (2sp a day each),
and three wait staff (2sp a day each), for a total of 8gp a day.\par

I listed the Smithy and the Alchemist as having a cost of 2sp a day,
because I imagined both of them as having apprentices.
Feel free to change that as needed.\par

\subsection{Different Locations}

To change a business to an upscale location, increase the daily cost by 50\%,
the paid-in capital by 50\%, and the profit die to $1d6$.
The competitiveness bonus remains the same.\par

Typically, cities are more profitable because of foot traffic,
and major junctions are more profitable.
Use the following table to decide on a location and the profit die.

Use the following table as a guide to determine the profit die and the cost of rent based on the location.

\begin{DndTable}[header=Location]{Xrr}
	Location	&	Profit Die \\
	Small Village	&	$1d4$ \\
	Town / Crossroads	&	$1d6$ \\
	City	&	$1d8$ \\
	Major City / Capital & $1d10$ \\
	Major Trade Hub & $1d12$ \\
\end{DndTable}

Major trade hubs are places like Waterdeep, Baldur's Gate, or Neverwinter in Forgotten Realms,
or Sharn in Eberron. There is supposed to just two or three of these in a typical campaign setting.

\subsection{Example Businesses}

% TODO: Smithy

% TODO: Bakery

% TODO: Magic Shop

% TODO: Alchemist

% TODO: Tavern

% TODO: Restaurant

% TODO: Market

% TODO: Large Market

% TODO: Large Competitor

\subsection{Empty Business Sheet}

\begin{DndMonster}[width=.5\textwidth - 8pt]{Name:}
	\DndMonsterType{Construct (Business), lawful evil}
	% Actions
	\DndMonsterSection{Stats}
	\DndMonsterAction{Daily Cost}
	\hfill ......gp
	\DndMonsterAction{Paid-in Capital}
	\hfill ......gp
	\DndMonsterAction{Competitiveness}
	\hfill ........
	\DndMonsterAction{Profit Roll}
	\hfill ........
	\DndMonsterAction{Required Proficiency}
	\hfill ........


\end{DndMonster}


\subsection{NPC Events \& Hooks}


Use the following either as a random table whenever you want to introduce a new encounter, or as inspiration to connect to adventures.
If the NPC working for the shop is gone, the cost per day drops by 1 gp, but the shop does not bring in revenue unless the player steps in or finds a replacement.
If the shipments do not come in, the shop cannot generate any revenue.\par

\begin{DndTable}[header=Events]{rX}
	Roll	&	Event or Hook \\
	1	&	The NPC working for the shop disappears. (They are kidnapped.)\\
	2	&	The NPC working for the shop rolls their relevant skill check against DC 20. If they win, they find another job and leave immediately.\\
	3	&	The NPC working for the shop leaves abruptly. They have a note saying that they are going to a relative's funeral, and will be back next month. Roll $1d4$, if the result is 1, they do not come back, otherwise, they come back the next month.\\
	4	&	The NPC working for the shop rolls Deception (Charisma) against the PP of the player. If they succeed, tell the player that the revenue is 1 less than the ``revenue die''. They will try again next month until they fail. If they fail, the shop owner catches them. \\
	5	&	Increased competition --- either a new store or an existing store engaging in a price war. The DC is increased by 5, and will remain so unless competition is removed.\\
	6	&	If there is increased competition, it ends. Otherwise, nothing happens.\\
\end{DndTable}


\subsection{Playtest Notes}

I am still actively developing this mechanic, and testing it in a campaign. I would love to hear ideas and feedback.\par

\section{Networking}

\begin{emphasisParagraph}
	Networking is to answer two important questions:
	Who do you know, and who knows you?
\end{emphasisParagraph}

When faced with a new NPC, a player can always ask if they know them, or somebody who knows them.
Similarly, if they go to a new location, they can ask if they know anybody there, or somebody who knows somebody there.
It may break immersion if the response is always a ``no''.
To avoid this, we let the player roll to see if they know the NPC, or somebody who knows them.
This is a Charisma check.
The DC is based on the location of the player and the organizations that they are a member of.
Here is the DC table:

\begin{DndTable}[header=Networking DC]{Xr}
\end{DndTable}
% TODO: Roll Cha to find out if people remember you. DC based on time passed.

% TODO: Network as an NPC

\section{Battle Preparation}
\label{sec:battle_preparation}

In almost all of the D\&D sessions that I played in various settings, the party went in blind to boss battles.
This is in contrast with a lot of what I see in fantasy literature.
Witchers seem to have a lot of theoretical and practical knowledge about beasts.
In the Witcher games, you have to pick the right potions, and get the right ``signs'' to use based on the monster.
In multiple computer games based on various editions of the D\&D rules, you would need potions or spells based on specific vulnerabilities.
In almost all of the computer games with boss battles, you have to save and load multiple times before coming up with a strategy.
What I am trying to get at is: preparing for the battle can be part the fun.\par

\begin{emphasisParagraph}
	In the novels and the game, Geralt of Rivia always prepares for a monster hunt.
	He reads up on the monster, prepares potions, oils and bombs,
	and sometimes even sets up traps.
	He does not just go in blind.
\end{emphasisParagraph}

\paragraph*{Cue the players}
As GM, consider using the following sentence: ``you will all die if you go into the battle without preparation''.
If possible, have an NPC make this very clear, or if you can, kill an NPC that they know to be strong and powerful.
Honestly prepare a boss that is twice as powerful as usual, or use the mechanic from \nameref{chap:d20_real_combat}.\par

\begin{emphasisParagraph}
	D\&D 5e rules have everything you need to make a battle deadly when you don't know what you are facing,
	and a deadly battle is a great motivator to prepare.
	Put in a strong boss, perhaps let them escape once, or make it clear that they will die if they go in unprepared.
	The strong boss is much easier to deal with when you know its vulnerabilities,
	or if you can set up a trap.
\end{emphasisParagraph}

\paragraph*{Doing Research}

To do research about a creature, one player in the party roll Investigation (Intelligence).
The DC is 20 + CR of the creature.
Upon a success, reveal to the players one of the following: An immmunity, an resistance, a vulnerability or a special attack.
(Optionally, you can extend this to learning a demon's true name, either the location or the description of a lich's phylactery.)
To do research, there has to be a source of information, such as a library, a wise person, or even just the tavern.
At each source, the party can roll only once, if they fail, the source does not have information relevant to the creature.\par

Typical sources of information are: taverns, markets, schools, colleges, guild halls, university campuses, wise hermits, and of course, libraries.
Particularly good sources of information will give a bonus die, see the table below.
However, research takes time, and the party may have to make a choice between facing the creature, or going at it alone.\par


\begin{DndTable}[header=Events]{rXX}
	Source					&	Bonus	& Time Spent \\
	Inn 					&	---		& 1d4 hours  \\
	Market 					&	---		& 1d4 hours  \\
	Town Square				&	---		& 1d4 hours  \\
	College (Professor) 	&	$1d4$	& 1 day  \\
	Library 				&	$1d6$	& 1 day  \\
	Arcane/Major Library 	&	$1d8$	& 1 week  \\
	Internet	 			&	$1d10$	& 1 day  \\
	Forbbidden Tomes		&	$1d12$	& 1 day  \\
	Generative AI			&	$1d12$	& 1 hour  \\
\end{DndTable}

If a location qualifies as both types of sources, choose the one with the higher bonus and do not roll more than once.\par

At DM's discretion, with ``Forbidden tomes'' and ``Generative AI'', if the bonus die is 1, there may be unpleasant circumstances.\par

So for example, Kaer Morhen, before it gets sacked, would probably have given \par

The check does not benefit from \emph{Bless} or \emph{Bardic Inspiration}, since the time scale is larger than the duration of these effects.
However, the expertise dice from earlier in this book may apply and stacks. (\nameref{chap:d20_plus_d4})\par


\begin{emphasisParagraph}
	Mechanics in the current section are for everything that can happen ahead of time before a battle.
	For a mechanic concerning the seconds before a violent encounter starts, see the \nameref{sec:dramatic_standoff} Section.\par
\end{emphasisParagraph}

\subsection{Examples}

\subsection{Playtest Notes}



\section{Financial Management \& Investment}
% Start with the Dutch tulip market craze
% TODO: Medieval era trading investments, like Janissaries or Arabian Nights.
% TODO: "safe" vs "risky" investments
% TODO: Modern era market investments.


\section{Social Endeavors}
\subsection{Making An Impression}
\subsection{Public Speeches}
\subsection{Dating}

\chapter{Conflict Against NPCs}
\section{Chases}

The D\&D has an excellent mechanic for chases.
You can take a dash action as many times as 3 + your Constitution modifier.
After that, each time you take a dash action, you must make a Constitution check against DC 10, and lose one level of exhaustion on failure.
If the quarry is outside the sight of the pursuers, they escape.\par

This mechanic, combined with the usual combat mechanics, create great chase scenes.
Make sure to have an enemy that is slightly faster than the group, perhaps using a \emph{Haste} spell.
In response, players may cast spells of their own, such as \emph{Grease}, \emph{Web} or \emph{Hold Person}.
Alternatively, they can attempt to chase the quarry as far as their Constitution allows.\par

If using maps, use a large city map rather than a detailed battle map.
Try to create a headstart for a quarry to give characters a challenge, and prefer high speed but low constitution monsters.\par

All of this should create enough entertainment for a chase scene.
If required, you can add random events that turn the terrain in front of the characters to difficult terrain, such as apples falling from a cart.\par



% TODO: Explain aerial combat specific issues

\subsection{Aerial Chases}
% TODO: Give the example of the aerial combat from my game.
\section{Racing}
\section{Sports \& Competitions}

% TODO: Otso Borno from this link, it is like Sumo: https://www.reddit.com/r/dndnext/comments/4jr6ly/comment/d39842b/
% However, I want to turn this into a coordination game: you choose STR or DEX. If you choose DEX when opponent chooses STR, you roll at an advantage. But if you both choose DEX or STR, a penalty happens, or something like that. You should also have an optional "rolling for insight" rule.
% Optional rule: Make it a drinking game.
% Optional rule: need three consecutive strikes...

% TODO: Billiards https://www.dndbeyond.com/forums/d-d-beyond-general/story-lore/22670-in-game-minigame-ideas

\section{Gambling \& Tavern Games}

There are many homebrew resources for tavern games.
The games here are based on the excleent Reddit post by u/eryan64, ``\href{https://www.reddit.com/r/DnDBehindTheScreen/comments/fn6tng/a_collection_of_tavern_games/}{A Collection of Tavern Games}''.
I am striving to create realism, and give the player a choice.

\subsection{Blackjack}
% https://www.reddit.com/r/DnDBehindTheScreen/comments/fn6tng/a_collection_of_tavern_games/

\paragraph{Cheating}
\subsection{Slots}

\paragraph{Slots: Realistic Odds and Payoffs}

\subsection{Gathering Information while Playing}

\subsection{Dice Poker}

% TODO: Dice poker from The Witcher https://witcher.fandom.com/wiki/The_Witcher_dice_poker

\subsection{Drinking Games}

\section{Dramatic Stand-Offs}
\label{sec:dramatic_standoff}

% TODO: Rewrite in a simplified manner. Add a flowchart.

Encounters do not always need to end in violence.
In the Western genre, there is oftentimes a ``standoff'' or ``showdown'' right before a gun battle.
This is a chance for one of the parties to back off, intimidated by their odds.
You could resolve such a standoff as a simple intimidation roll, or you could turn it into a set of decisions.\par

\begin{emphasisParagraph}
	Encounters do not always need to end in violence.
	The traditional course of action is to have NPCs attack, typically against the odds, and leave no choice to the players.
	A realistic and dramatic alternative is to have a stand-off: the characters can use a few rounds to assess the odds,
	try to intimidate, prepare and escalate, or try to de-escalate.
	This ``stand-off'' before the battle can be used for assessing the opponents' strengths and weaknesses,
	stealthily manouvering, or stealthily casting spells.
\end{emphasisParagraph}

\paragraph*{Readying Weapons}
Just like in the Western movie trope, the characters can choose to brandish or ready their weapons.
If they brtandish their weapons, this gives them an advantage in Intimidation (Cha) rolls.
I suggest 20 + CR of the highest ranking enemy + number of enemies as the DC or 10 + total HD of the enemies.
If the roll fails after the characters brandish their weapons, the fight will begin and everyone has to resort to violence.\par

Alternatively, the characters may try to ready their weapons secretly.
Have them roll Sleight of Hand (Dex) against the closest enemy.
If they pass, they gain an advantage on the Initiative roll.
If they fail, they still gain the advantage, but the enemies will also ready their weapons and they will also gain the same advantage.\par

\paragraph*{Intimidation}
Finally, characters can try to escalate the situation through Intimidation (Cha) or de-escalate through Persuasion (Cha).
Ask for three successes on Intimidation (Cha) rolls, just like the death saving throw mechanic.
This is to make sure that the stand-off is fun, similar to the violence mechanic.
If the first roll fails, the situation automatically decsends into violence.
After that, either three fails, or one natural 1 roll will cause the battle to start.
Exact nature of success depends on the storyline:
upon three successes, either the entire team of enemies disband, or the henchmen or NPCs with little skin inthe game will run away, or some of them will yield and accept terms.\par

\paragraph*{De-escalation}
An alternative is to try and de-escalate.
Require three successes on Persuasion (Cha) rolls.
If the characters brandished their weapons or readied themselves for combat, the rolls are at a disadvantage, up until they lower their weapons.
A natural 1 causes the battle to start - note that if their weapons are lowered, and the opponents are ready, the players have disadvantage for the initiative roll.
\par

\paragraph*{Studying Opponents}
As the stand-off goes on, players can use this time to assess their opponents' weaknesses and strengths.
Depending on the type of the adversary, ask for Arcana (Int), Animal Handling (Wis), Insight (Wis) or Investigate (Int) checks to understand each immunity, resistance, vulnerability and actions of the adversary.
The suggested DC for these rolls is DC 15 + CR of the creature.\par

\paragraph*{Stealth Action \& Manuvering In Place}
The stand-off is also an opportunity to try and cast spells such as \emph{Charm} or \emph{Hold Person}, assuming that the player can cast these in a stealthy manner.
For example, a sorceress can remain behind and Hide (Dex) to try and cast either one of these spells.
If the NPCs realize that they are being tricked in this way, the encounter descends into violence.\par

\begin{emphasisParagraph}
	Why would a troupe prefer to have a stand-off before battles?
	Stand-offs add drama. Stand-offs are realistic.
	Stand-offs increase the tension by using uncertainty.
	Stand-offs give chance to characters with scores other than Strength, Dexterity and Constitution to contribute.
	They are good for parties with a varied set of skills, and for players that are bored from the usual combat mechanic.\par
\end{emphasisParagraph}

\section{Hauntings}
% TODO: Combine with Exorcism below.

\section{Exorcism}

In D\&D 5th Edition, there are no gewneralized rules for possession.
In the case of possession by a ghost, the victim needs to succeed at a Charisma save.
Otherwise, a spell of \emph{Dispel Evil} can end the possession.
There are no rules for ending possessions from other beings, such as fiends, outsiders, or more powerful ethereal undead.
This mechanic is a general combat mechanic for ending possession.\par

% TODO: Paraphrase passage below.

Require a total of spell attacks equal to the hit dice of the possessing entity or monster.
On any failure, the spellcaster takes one level of exhaustion. Instead of death, the possessing entity gains control of the character.
Limit this ability to only Wisdom-based spellcasters. % TODO: Maybe add a rule for Persuasion rolls from loved ones?
Attack rolls are made with advantage if any one of the following is used:
\begin{itemize}
	\item The player knows their true name.
	\item The player has a personal object of a ghost.
	\item The player has a divine spell focus.
	\item The player has an additional item with divine quality, such as holy water.
\end{itemize}

Each one of these items can be used once to get an advantage once during an exorcism session.
Multiple personal objects of a ghost can be used once, each.

At the end of an exorcism, the possessing entity will appear next to the victim, and can be attacked normally.\par

The purpose of this mechanic is give incentive to players to prepare for such an encounter.\par


% TODO: Cite "Ultimate Bestiary The Dreaded Accursed" for more possessing entities.
\subsection{Subterfuge}

% TODO: Add rule for false appearance: The difficulty to spot is DC 20 + the monster's CR or the spellcaster's spellcasting abilty.

\chapter{Conflict Against Nature}
\section{Firefighting \& Spread Mechanic}

In D\&D, we have Fire Elementals, a monster.
In real life, fire itself is a threat, and a worse one.
Fire is a monster.\par


\begin{DndMonster}[width=.5\textwidth - 8pt]{Fire}
	\DndMonsterType{Elemental, neutral}

	\DndMonsterBasics[
		armor-class = {12 (natural armor)},
		hit-points  = {\DndDice{1d10}},
		speed       = {0 ft.},
	]

	\DndMonsterAbilityScores[
		str = 12,
		dex = 0,
		con = 10,
		int = 0,
		wis = 0,
		cha = 0,
	]

	\DndMonsterDetails[
		damage-vulnerabilities = {cold, water, suffocation},
		damage-immunities  = {bludgeoning, piercing, slashing, fire, magic, poison, necrotic, acid, poison, radiant, psychic},
		damage-resistances   = {thunder, force},
		condition-immunities = {charmed, frightened, grappled, paralyzed, prone, poisoned, stunned, unconscious},
		senses = {none},
		languages = --,
		challenge = 1/8,
	]

	% Actions
	\DndMonsterSection{Actions}

	\DndMonsterMelee[
		name=Burn,
		mod=+1,
		dmg=\DndDice{1d10+1},
		dmg-type=fire,
		%extra=,
	]

	\DndMonsterAction{Spread}
	At each succesful hit, the fire either adds another hit die to itself, or

	\DndMonsterSection{Description}
	Fire has no senses, and is immune all damage types other than cold, thunder and force.
	It has resistance to thunder and force.
	It is immune to all conditions other than invisible and petrified.
	It still emits heat when invisible, but petrification puts it out.
	It makes attack only within 5ft, choosing the victim randomly.
	Whenever it succeeds at an attack roll, it increases by one hit die, or clones itself to the next square.
	If there are materials vulnerable to fire, it will spread to these squares first.
	In addition, any magical or non-magical effect that causes suffocation will do double damage.

\end{DndMonster}

With this mechanic, fire is very easy to deal with if the players have the right materials and if they address it early.
If they do not have the right materials, or if they let the fire spread, it can turn into a much more powerful monster within five to ten turns.
One possibility is to have the fire going on a battlemap as a battle goes on.
The players may choose to take advantage of the fire, and decide to deal with it, or may choose to spread their effort between enemy combatants and the fire.\par

\subsection{Spread Mechanic}

This type of spread-on-hit mechanic can be extended to other threats, such as oozes and biological threats.

\section{Floods}

\section{Climbing}
% TODO: Inpiration from the scenes in Damsel
% https://rpg.stackexchange.com/questions/103859/how-do-i-design-a-climb-up-a-cliff-challenge

% TODO: Full Contact Tree Climbing https://www.reddit.com/r/DnD/comments/f33095/what_minigames_do_you_use/



\section{River Crossing}
% TODO: Wagon River Crossing https://www.reddit.com/r/DnD/comments/f33095/what_minigames_do_you_use/


\section{Navigation}

% TODO: Basic idea: roll survival to offset penalties to speed, or roll survival to open up other areas on the map.

\chapter{Conflict Against Society}
\section{Homelessness}
\section{Cities as Monsters}
% TODO: Write the stat blocks as a gargantuan swarm.
\section{Governments as Monsters}
\section{Inflation}



\chapter{Conflict Against the Supernatural}
\section{Conflict Against the Fey}
% Refer to https://the-eye.eu/public/Books/rpg.rem.uz/Dungeons%20%26%20Dragons/3rd%20Party/5th%20Edition/Legendary%20Games/Faerie%20Passions.pdf

\section{Conflict Against Cosmic Forces}

\section{Resisting Possession}

% TODO: Same as exorcism, but the person themselves can do it even if they are not a Wisdom-based spellcaster.


\chapter{Internal Conflict}
\section{Anxiety}
% TODO: Write a stat block as a monster, immune to all damage. Casts Fear as a spell, once per day. Calm can help. Heroism should be fixing it, you should also have a saving throw, and the anxiety can have levels as a spellcaster.
% https://www.reddit.com/r/dndnext/comments/bar5iy/how_to_counter_fear_andor_paralyze/

\begin{DndMonster}[width=.5\textwidth - 8pt]{Anxiety}
	\DndMonsterType{Fiend, chaotic neutral}

	\DndMonsterBasics[
		armor-class = {-},
		hit-points  = {\DndDice{1d10}},
		speed       = {0 ft.},
	]

	\DndMonsterAbilityScores[
		str = 0,
		dex = 0,
		con = 0,
		int = 0,
		wis = 0,
		cha = 12,
	]

	\DndMonsterDetails[
		damage-vulnerabilities = {psychic, radiant},
		damage-immunities = {bludgeoning, cold, suffocation, piercing, slashing, fire, magic, poison, necrotic, acid, poison, thunder, force},
		senses = {none},
		languages = --,
		challenge = 1/8,
	]

	\DndMonsterSection{Actions}
	\DndMonsterAction{Panic Attack}

	\DndMonsterSection{Traits}
	\DndMonsterAction{Damage Transfer}
	When attacked with psychic or radiant damage, the player takes only half the psyhic damage dealt to it, anxiety takes the other half.

	\DndMonsterAction{Damage Transfer}
	If the player is a Charisma-based spellcaster, they can opt to use thier Anxiety's Charisma instead of their own.
	If they do so, on a critical hit, the player suffers from a panic attack.

	\DndMonsterSection{Description}
	Anxiety is only in the hosts' mind, so it has no physical attributes, or speed.
	It goes wherever the host goes.
	For Intelligence and Wisdom, it uses the host's scores.
	However, it has its own Charisma, which determines how strong it is.

	\DndMonsterSection{Roleplay Mechanics}
	\DndMonsterAction{What If?}
	When a character has anxiety, the player has to come up with ideas about what could go wrong if the roll fails.

\end{DndMonster}




\section{Insecurity}
\section{Curses}
% TODO: How to overcome curses
% TODO: Specific curses. Trigger when a number is rolled, for instance, 13?
\section{Sickness}
\section{Addiction}
\subsection{Alcoholism}

\section{Other Demons}
% TODO: Consider the madness effects on page 80 from Fraternity of Shadows - Ravenloft - Heroes of the Mists
\subsection{Obsession \& Paranoia}
% TODO: To roll-play obsession, have the player roll investigation checks on the objects of obsession. If they fail, tell them that they do not find anything.
% TODO: To roll-play paranoia, have the player roll perception checks. If they fail, tell them that nobody is following them. Alternatively, tell ask for insight checks in social scenarios, and on failures, tell them that there are no plots against them.


\begin{emphasisParagraph}
	Fantasy demons can become meaningful to players if they subtly symbolize real life demons.
\end{emphasisParagraph}

This is a simple mechanic: As a GM, generate an NPC demon.
Give the demon an agenda, and keep it hidden from the player.
This demon will occasionally communicate with the player.
In fact, it's better if the player summons a demon and decides to enlist their help.\par

If an action that the player is taking is contributing to the demon's agenda, the player will get a bonus die of $d4$, $d6$, $d8$, $d10$ or $d12$.
(Get the demon's hit dice, round up to the nearest die.)\par
As the GM, you decide whether the player gets the additional die. Make sure that the player knows the bonus die, and in fact, make them roll it.
This die also stacks with \emph{Bless} and Bardic Inspiration.\par

The purpose of this mechanic is to build tension and suspense. It should be occasional, and over time in a longer campaign, the player will finally triangulate what the demon's agenda is.\par

\subsection{Optional Rule: Demon Gets Stronger}
Track every time a Nat 20 is rolled on the rolls where the demon bonus applied.
This is the number of times that the demon can cast \emph{Hold Person} at will on the said player.
Cue the player either before or after the first nat 20 roll.
This is important to let them know.
Once the player is aware that this is giving the demon some sort of control over them, they can choose to not apply the bonus.

Then put them in desperate situations where they need the bonus.\par


% TODO: Add a list of possible agendas, or refer to Fishel 2023.

\subsection{Playtest Notes}

% TODO: Add


\chapter{Spellcasting}
\section{Learning Spells}

\subsection{Conjuration Spells}

% TODO: The player can learn spells by defeating elementals and aberrations. Give a template for each level, explain the filter and restriction. They can also combine this with the Stand-Off mechanic.
% TODO: Optional rule: Add a check to control the conjured creature, if the total level is higher than the caster. Or, add the levels of the creatures and roll against that level. Or have each creture roll a Wisdom saving throw against the caster.
% Cite the ``Kobold Guide to Magic''



\subsection{Playtest Notes}

In a game, one of the characters (bard) tried to communicate with a fire mephit just before a fight.
I ruled that the attempt has failed, since mephits are not very intelligent.
However, to reward them, I gave them a (homebrew) second-level conjuration spell for fire, magma and steam mephits (they all speak Ignan).
The interpretation is that they understood how to communicate with these elementals.\par

\section{Player Creativity in Spellcasting}

\begin{emphasisParagraph}
	Use this if you want to allow for some creativity in spellcasting (without breaking the balance).
\end{emphasisParagraph}

\begin{emphasisParagraph}
	How many times have players asked you whether they can light up the grease from the \emph{Grease} spell?
	Typically, the DM has to say no. This section suggests a way for allowing for player creativity while keeping it balanced.
\end{emphasisParagraph}

D\&D uses the ``Vancian'' spellcasting paradigm: there is a known recipe with a known effect.
If you follow the recipe, you get the effect.
There is no written way to get a slightly different effect in the rules.\par

Other systems, such as Mage the Ascension in the Chronicles of Darkness, allow for limitless creativity.
This is fun, but can make things difficult for the GM.\par

The rules in this chapter are for getting the best of both worlds.
As a GM, you do not refuse the creative players, but you do make things challenging, so that you do not lose control over the texture of reality.

\subsection{Metamagic: ``Nudge'' the Spell}

The spellcaster wants to change one aspect of the spell, without changing the damage, level, or duration.
For example, the damage type may change.
The area of an area effect spell might change.
Alternatively, what the spell does can change.
This should not be a change that gives an absolute advantage in mechanical terms.\par

In this case, the player rolls Intelligence (Arcana) DC 15 + the level of the spell.
If they fail, the spell fizzles with no effect.
If they roll a nat 1, the spell ``backfires'' in some way.
Optionally, if they roll a nat 20, they can add the spell as a new spell.\par

% TODO: A wonderful example comes from ``Dimension 20'' with the web spell. Add here.

\subsection{Metamagic: Make the Spell More Powerful}
The spellcaster wants to cast the spell at a higher level and increase the damage or duration accordingly.
First decide on what the level of the spell would be. For example, if they want \emph{Fireball} to deal as much damage as \emph{Cone of Cold} (8d8), the new spell level is 5.
The player rolls Intelligence (Arcana) DC 15 + the new level of the spell as determined by the DM.
If they fail, they lose the slot or the memorized spell.
If the new spell level is a level that they do not have access to, also ask for a Constitution saving throw, DC 12 + the new level.
If they fail, they take one level of exhaustion.



\section{Dark Magic}

\subsection{Blood Magic}

\subsection{Shadow Magic}

\subsection{Mirror Magic}

\section{Alternative Rules}

\subsection{Magic Takes Its Toll}

\subsection{Permanency}

% TODO: Why this rule?
% TODO: Mention the rules for permanency in PHB.
% TODO: Explain how you hated as a player to see NPCs have access to powers that you do not have access to.

Repeat the spell, at the same location. With each repetition, roll an Intelligence (Arcana) check against DC 20 + the spell's level.
\begin{enumerate}
	\item If they accumulate as many successes as the spell's level + 1, the effect becomes permanent.
	\item If they roll a nat 1, they reset.
	\item If they roll a nat 20, the effect becomes permanent.
	\item If they cast another spell, they reset.
\end{enumerate}

The player can rest in between to replenish their spells.

The spell must have a duration.\par

Yes, this would be one way for players to create an undead army.
I always wanted to find out how to do that, as a player.\par

% TODO: Use this to give conjured demons personality and turn them into servants...
% Cite the ``Kobold Guide to Magic''


% TODO: Increasing the impact or area of a spell
% TODO: Joint spellcasting
% TODO: Adding stats to creations
% TODO: Not using components
% TODO: Wild magic backlashes per spell



% TODO: _Awakened_ animals as races
% TODO: _Awakened_ modern objects as races
% TODO: Roleplaying childhood and teenage years

\chapter{Rewarding Creativity}

% Good but futile idea: 25
% Good idea that moved the party forward but created troubles: 50
% Great idea that worked: 100
% Tactical use of spells: divide XP of the monster by the spell level.
% Successful deescalation: XP level
% Attempt at deescalation that failed: XP level / 4
% Party-level achivement: 25 / 50 / 100
% City/County-level achivement: 250 / 500 / 1000
% Country/Plane-level achivement: 2500 / 5000 / 10000
% Continent/Planet/Plane-level achivement4: 25000 / 50000 / 100000


% Reward for achieving goals?

% \chapter{Crafting}
% \section{Alchemy}
% TODO: Potions, Poisons and Bombs
% TODO: Take inspiration from City & Wild, Alchemy section. Simplify and convert to Eberron, Forgetten Realms or other settings. Each plane represents one damage type.

% TODO: Let's follow the cocktail approach: three elements: base, an "essence" based on the planar influence and something else...

% \section{Metalworking: Weapons /& Armor}

% \section{Enchanting}


\chapter*{Suggested Reading}

% TODO: Fix the format

\cite{Fishel2023} % Proactive Roleplaying
% Eberron Linguistics


\printbibliography[heading=none]

\end{document}
